\documentclass[a4,11pt]{article}
\usepackage{geometry}
\geometry{a4paper, landscape, top=20mm, left=20mm, right=20mm, bottom=15mm}
\pagenumbering{gobble}
\setlength\parindent{0pt}
\usepackage[german]{babel}
\usepackage[utf8]{inputenc}


\usepackage{multicol} % Spalten
\usepackage{color} % Farben
\usepackage[hyphens]{url} % Internetadresse (mit automatischer Trennung)
\usepackage{enumitem} % Aufzählungen




% Mathesymbole
\usepackage{amsmath, amsthm, amssymb}
\usepackage{bm} % fette Schrift in Matheumgebung


% Bilder
\usepackage{caption}
\usepackage{graphicx, wrapfig}
\usepackage{subcaption} % Bilder in Gruppe einzeln benennen

\usepackage[
labelfont=bf,        % Tabelle x: Abbildung y: ist jetzt fett
font=small,          % Schrift etwas kleiner als Dokument
width=0.9\textwidth, % maximale Breite einer Caption schmaler
]{caption}

\begin{document}
\section{Operatoren}
Operatoren aus Chiral Effective Theory heißen R,
\begin{align*}
	R_{1q}^{(6)} &= (\bar{\chi}\gamma_\mu\chi)(\bar{q}\gamma^\mu q) \\
	R_{2q}^{(6)} &= (\bar{\chi}\gamma_\mu\gamma_5\chi)(\bar{q}\gamma^\mu q) \\
	R_{3q}^{(6)} &= (\bar{\chi}\gamma_\mu\chi)(\bar{q}\gamma^\mu\gamma_5q) \\
	R_{4q}^{(6)} &= (\bar{\chi}\gamma_\mu\gamma_5\chi)(\bar{q}\gamma^\mu\gamma_5q)
\end{align*}
die anderen Q
\begin{align*}
	Q_{1ij}^{(6)} &= (\bar{\chi}\gamma_\mu\tilde{\tau}^a\chi)(\bar{Q}_L^i\gamma^\mu \tau^aQ_L^j) \\
	Q_{2ij}^{(6)} &= (\bar{\chi}\gamma_\mu\chi)(\bar{Q}_L^i\gamma^\mu Q_L^j) \\
	Q_{3ij}^{(6)} &= (\bar{\chi}\gamma_\mu\chi)(\bar{u}_R^i\gamma^\mu u_R^j) \\
	Q_{4ij}^{(6)} &= (\bar{\chi}\gamma_\mu\chi)(\bar{d}_R^i\gamma^\mu d_R^j) \\
	Q_{5ij}^{(6)} &= (\bar{\chi}\gamma_\mu\gamma_5\tilde{\tau}^a\chi)(\bar{Q}_L^i\gamma^\mu \tau^aQ_L^j) \\
	Q_{6ij}^{(6)} &= (\bar{\chi}\gamma_\mu\gamma_5\chi)(\bar{Q}_L^i\gamma^\mu Q_L^j) \\
	Q_{7ij}^{(6)} &= (\bar{\chi}\gamma_\mu\gamma_5\chi)(\bar{u}_R^i\gamma^\mu u_R^j) \\
	Q_{8ij}^{(6)} &= (\bar{\chi}\gamma_\mu\gamma_5\chi)(\bar{d}_R^i\gamma^\mu d_R^j)
\end{align*}
Das heißt es werden 12 (3 Flavour x 4 Operatoren) Koeffizienten auf 72 (8 Operatoren x 3x3 Möglichkeiten) Koeffizienten aufgeblasen, wobei sich die 72 auf 42 reduzieren durch die Annahmen \textit{nur u,d,s} und \textit{keine Mischterme}.
\section{Kürzen der $Q^{(6)}$: Nur u,d,s und diagonale Terme}
$Q_{1ij}^{(6)},Q_{2ij}^{(6)},Q_{5ij}^{(6)},Q_{6ij}^{(6)}$ können zusammen ausgerechnet werden, mit $q_L^{id} = V_{id}d_L+V_{is}s_L+V_{ib}b_L$:
\begin{align*}
	Q_L^iAQ_L^j &= \begin{pmatrix}
		\bar{q}_L^{iu} \\ \bar{q}_L^{id}
	\end{pmatrix}
	A \begin{pmatrix}
		q_L^{ju} \\ q_L^{jd}
	\end{pmatrix} \\
	&= \bar{q}_L^{iu}Aq_L^{ju} + \bar{q}_L^{id}Aq_L^{jd} \\
	&= \bar{q}_L^{iu}Aq_L^{ju} + (V_{id}^*\bar{d}_L + V_{is}^*\bar{s}_L+V_{ib}^*\bar{b}_L)A(V_{jd}d_L+V_{js}s_L+V_{jb}b_L) \\
	\tilde{Q}_1 = \tilde{Q}_5 &\approx \bar{q}_L^{iu}\gamma^\mu \frac{1}{2}q_L^{ju}\delta_{3a}
	+ (V_{id}^*\bar{d}_L + V_{is}^*\bar{s}_L+V_{ib}^*\bar{b}_L)\gamma^\mu \frac{-1}{2}(V_{jd}d_L+V_{js}s_L+V_{jb}b_L)\delta_{3a} \\
	&\approx \bar{u}_L\gamma^\mu \frac{1}{2}u_L\delta_{ij}\delta_{iu}\delta_{3a}
	 - \frac{1}{2}(V_{id}^*V_{jd}\bar{d}_L\gamma^\mu d_L + V_{is}^*V_{js}\bar{s}_L\gamma^\mu s_L)\delta_{3a} \\
%
	\tilde{Q}_2 = \tilde{Q}_6 &\approx \bar{q}_L^{iu}\gamma_\mu q_L^{ju}\delta_{ij} + V_{id}^*V_{jd}\bar{d}_L\gamma_\mu d_L + V_{is}^*V_{js}\bar{s}_L\gamma_\mu s_L \\
	&= \bar{u}_L\gamma_\mu u_L\delta_{iu}\delta_{ij} + V_{id}^*V_{jd}\bar{d}_L\gamma_\mu d_L + V_{is}^*V_{js}\bar{s}_L\gamma_\mu s_L \\
	\tilde{Q}_3 = \tilde{Q}_7 &= \bar{q}_R^{iu}\gamma^\mu q_R^{ju} \\
	&\approx \bar{q}_R^{iu}\gamma^\mu q_R^{ju}\delta_{ij} \\
	&= \bar{u}_R\gamma^\mu u_R\delta_{ij}\delta_{iu} \\
	\tilde{Q}_4 = \tilde{Q}_8 &= \bar{q}_R^{id}\gamma^\mu q_R^{jd} \\
	&\approx \bar{d}_R\gamma^\mu d_R\delta_{ij}\delta_{id} + \bar{s}_R\gamma^\mu s_R\delta_{ij}\delta_{is}
\end{align*}
Das heißt, dass einige der Koeffizienten automatisch 0 sind. Nämlich alle
\begin{align*}
	&C_{3ij} \quad\text{außer}\quad C_{311} \\
	&C_{4ij} \quad\text{außer}\quad C_{411},C_{422} \\
	&C_{7ij} \quad\text{außer}\quad C_{711} \\
	&C_{8ij} \quad\text{außer}\quad C_{811},C_{822} \quad.
\end{align*}
Für den Vorfaktor gilt
\begin{align*}
	(\bar{\chi}A\tilde{\tau}^a\chi) = \tau_{0}(\bar{\chi}_0A\chi_0)
\end{align*}
mit $\tau_{0}=0$, falls $a$ nicht der richtige Index ist.
\section{Umschreiben der $Q^{(6)}$ von $q_L,q_R$ in $q$}
\begin{align*}
	\tilde{Q}_1^{(6)} = \tilde{Q}_5^{(6)} &= \bar{u}\gamma^\mu \frac{1-\gamma_5}{4}u\delta_{ij}\delta_{iu}\delta_{3a}
	- V_{id}^*V_{jd}\bar{d}\gamma^\mu\frac{1-\gamma_5}{4} d\delta_{3a} - V_{is}^*V_{js}\bar{s}\gamma^\mu\frac{1-\gamma_5}{4} s\delta_{3a} \\
	&= \frac{1}{4}(\bar{u}\gamma^\mu u\delta_{ij}\delta_{iu}\delta_{3a}
	- V_{id}^*V_{jd}\bar{d}\gamma^\mu d\delta_{3a} - V_{is}^*V_{js}\bar{s}\gamma^\mu s\delta_{3a}) - \frac{1}{4} (\bar{u}\gamma^\mu \gamma_5u\delta_{ij}\delta_{iu}\delta_{3a}
	- V_{id}^*V_{jd}\bar{d}\gamma^\mu\gamma_5 d\delta_{3a} - V_{is}^*V_{js}\bar{s}\gamma^\mu\gamma_5 s\delta_{3a}) \\
	%%%%%%%%%%%%%%%%%%%%%%%%%%%%%%%%%%%%%%%%%%%%%%%%%%%%%%%%%%%%%%%%%%%%%%%%%%%%%%%%%%%%%%%%%%%%%%%%%%%%%%%%%%%%%%%%%%%%
	\tilde{Q}_2^{(6)} = \tilde{Q}_6^{(6)} &= \bar{u}\gamma^\mu\frac{1-\gamma_5}{2}u\delta_{iu}\delta_{ij} + V_{id}^*V_{jd}\bar{d}\gamma^\mu\frac{1-\gamma_5}{2}d + V_{is}^*V_{js}\bar{s}\gamma^\mu\frac{1-\gamma_5}{2}s \\
	&= \frac{1}{2}(
	\bar{u}\gamma^\mu u\delta_{iu}\delta_{ij}-\bar{u}\gamma^\mu\gamma_5u\delta_{iu}\delta_{ij}) + \frac{1}{2}(V_{id}^*V_{jd}\bar{d}\gamma^\mu d - V_{id}^*V_{jd}\bar{d}\gamma^\mu\gamma_5d
	+ V_{is}^*V_{js}\bar{s}\gamma^\mu s - V_{is}^*V_{js}\bar{s}\gamma^\mu\gamma_5s) \\
	%%%%%%%%%%%%%%%%%%%%%%%%%%%%%%%%%%%%%%%%%%%%%%%%%%%%%%%%%%%%%%%%%%%%%%%%%%%%%%%%%%%%%%%%%%%%%%%%%%%%%%%%%%%%%%%%%%%%
	\tilde{Q}_3^{(6)} = \tilde{Q}_7^{(6)} &= \bar{u}_R\gamma^\mu u_R\delta_{ij}\delta_{iu} \\
	&= \bar{u}\gamma^\mu \frac{1+\gamma_5}{2}u\delta_{ij}\delta_{iu} \\
	&= \frac{1}{2}(\bar{u}\gamma^\mu u\delta_{ij}\delta_{iu} + \bar{u}\gamma^\mu \gamma_5u\delta_{ij}\delta_{iu}) \\
	%%%%%%%%%%%%%%%%%%%%%%%%%%%%%%%%%%%%%%%%%%%%%%%%%%%%%%%%%%%%%%%%%%%%%%%%%%%%%%%%%%%%%%%%%%%%%%%%%%%%%%%%%%%%%%%%%%%%%
	\tilde{Q}_4^{(6)}  = \tilde{Q}_8^{(6)} &= \bar{d}_R\gamma^\mu d_R\delta_{ij}\delta_{id} + \bar{s}_R\gamma^\mu s_R\delta_{ij}\delta_{is} \\
	&= \bar{d}\gamma^\mu\frac{1+\gamma_5}{2}d\delta_{ij}\delta_{id} + \bar{s}\gamma^\mu \frac{1+\gamma_5}{2}s\delta_{ij}\delta_{is} \\
	&= \frac{1}{2}(\bar{d}\gamma^\mu d\delta_{ij}\delta_{id} + \bar{d}\gamma^\mu\gamma_5d\delta_{ij}\delta_{id} + \bar{s}\gamma^\mu s\delta_{ij}\delta_{is} + \bar{s}\gamma^\mu \gamma_5s\delta_{ij}\delta_{is})
\end{align*}
\section{Umschreiben der $Q^{(6)}$ in $R^{(6)}$}
\begin{align*}
	Q_{1ij}^{(6)} &= (\bar{\chi}_0\gamma_\mu\chi_0)\delta_{3a}\tau_{0}\frac{1}{4}(\bar{u}\gamma^\mu u\delta_{ij}\delta_{iu}
	- V_{id}^*V_{jd}\bar{d}\gamma^\mu d - V_{is}^*V_{js}\bar{s}\gamma^\mu s) - (\bar{\chi}_0\gamma_\mu\chi_0)\tau_{0}\frac{1}{4} (\bar{u}\gamma^\mu \gamma_5u\delta_{ij}\delta_{iu}
	- V_{id}^*V_{jd}\bar{d}\gamma^\mu\gamma_5 d - V_{is}^*V_{js}\bar{s}\gamma^\mu\gamma_5 s) \\
	&=\frac{\delta_{3a}\tau_{0}}{4}(R_{1u}^{(6)}\delta_{ij}\delta_{iu}
	- V_{id}^*V_{jd}R_{1d}^{(6)} - V_{is}^*V_{js}R_{1s}^{(6)}) -\frac{\delta_{3a}\tau_{0}}{4} (R_{3u}^{(6)}\delta_{ij}\delta_{iu}
	- V_{id}^*V_{jd}R_{3d}^{(6)} - V_{is}^*V_{js}R_{3s}^{(6)}) \\
	%%%%%%%%%%%%%%%%%%%%%%%%%%%%%%%%%%%%%%%%%%%%%%%%%%%%%%%%%%%%%%%%%%%%%%%%%%%%%%%%%%%%%%%%%%%%%%%%%%%%%%%%%%%%%%%%%%%%%%%%
	Q_{2ij}^{(6)} &= (\bar{\chi}\gamma_\mu\chi)\frac{1}{2}(
	\bar{u}\gamma_\mu u\delta_{iu}\delta_{ij}-\bar{u}\gamma_\mu\gamma_5u\delta_{iu}\delta_{ij}) + (\bar{\chi}\gamma_\mu\chi)\frac{1}{2}(V_{id}^*V_{jd}\bar{d}\gamma_\mu d - V_{id}^*V_{jd}\bar{d}\gamma_\mu\gamma_5d
	+ V_{is}^*V_{js}\bar{s}\gamma_\mu s - V_{is}^*V_{js}\bar{s}\gamma_\mu\gamma_5s) \\
	&= \frac{1}{2}(
	R_{1u}^{(6)}\delta_{iu}\delta_{ij}-R_{3u}^{(6)}\delta_{iu}\delta_{ij}) + \frac{1}{2}(V_{id}^*V_{jd}R_{1d}^{(6)} - V_{id}^*V_{jd}R_{3d}^{(6)}
	+ V_{is}^*V_{js}R_{1s}^{(6)} - V_{is}^*V_{js}R_{3s}^{(6)}) \\
	%%%%%%%%%%%%%%%%%%%%%%%%%%%%%%%%%%%%%%%%%%%%%%%%%%%%%%%%%%%%%%%%%%%%%%%%%%%%%%%%%%%%%%%%%%%%%%%%%%%%%%%%%%%%%%%%%%%%%%%
	Q_{3ij}^{(6)} &= (\bar{\chi}\gamma_\mu\chi)\frac{1}{2}(\bar{u}\gamma^\mu u\delta_{ij}\delta_{iu} + \bar{u}\gamma^\mu \gamma_5u\delta_{ij}\delta_{iu}) \\
	&= \frac{1}{2}(R_{1u}^{(6)}\delta_{ij}\delta_{iu} + R_{3u}^{(6)}\delta_{ij}\delta_{iu}) \\
	%%%%%%%%%%%%%%%%%%%%%%%%%%%%%%%%%%%%%%%%%%%%%%%%%%%%%%%%%%%%%%%%%%%%%%%%%%%%%%%%%%%%%%%%%%%%%%%%%%%%%%%%%%%%%%%%%%%%%%%
	Q_{4ij}^{(6)} &= (\bar{\chi}\gamma_\mu\chi)\frac{1}{2}(\bar{d}\gamma^\mu d\delta_{ij}\delta_{id} + \bar{d}\gamma^\mu\gamma_5d\delta_{ij}\delta_{id} + \bar{s}\gamma^\mu s\delta_{ij}\delta_{is} + \bar{s}\gamma^\mu \gamma_5s\delta_{ij}\delta_{is}) \\
	&= \frac{1}{2}(R_{1d}^{(6)}\delta_{ij}\delta_{id} + R_{3d}^{(6)}\delta_{ij}\delta_{id} + R_{1s}^{(6)}\delta_{ij}\delta_{is} + R_{3s}^{(6)}\delta_{ij}\delta_{is}) \\
	%%%%%%%%%%%%%%%%%%%%%%%%%%%%%%%%%%%%%%%%%%%%%%%%%%%%%%%%%%%%%%%%%%%%%%%%%%%%%%%%%%%%%%%%%%%%%%%%%%%%%%%%%%%%%%%%%%%%%%%
	Q_{5ij}^{(6)} &= (\bar{\chi}_0\gamma_\mu\chi_0)\delta_{3a}\tau_{0}\frac{1}{4}(\bar{u}\gamma^\mu u\delta_{ij}\delta_{iu}
	- V_{id}^*V_{jd}\bar{d}\gamma^\mu d - V_{is}^*V_{js}\bar{s}\gamma^\mu s) - (\bar{\chi}_0\gamma_\mu\chi_0)\delta_{3a}\tau_{0}\frac{1}{4} (\bar{u}\gamma^\mu \gamma_5u\delta_{ij}\delta_{iu}
	- V_{id}^*V_{jd}\bar{d}\gamma^\mu\gamma_5 d - V_{is}^*V_{js}\bar{s}\gamma^\mu\gamma_5 s) \\
	&=\frac{\delta_{3a}\tau_{0}}{4}(R_{2u}^{(6)}\delta_{ij}\delta_{iu}
	- V_{id}^*V_{jd}R_{2d}^{(6)} - V_{is}^*V_{js}R_{2s}^{(6)}) -\frac{\delta_{3a}\tau_{0}}{4} (R_{4u}^{(6)}\delta_{ij}\delta_{iu}
	- V_{id}^*V_{jd}R_{4d}^{(6)} - V_{is}^*V_{js}R_{4s}^{(6)}) \\
	%%%%%%%%%%%%%%%%%%%%%%%%%%%%%%%%%%%%%%%%%%%%%%%%%%%%%%%%%%%%%%%%%%%%%%%%%%%%%%%%%%%%%%%%%%%%%%%%%%%%%%%%%%%%%%%%%%%%%%%%
	Q_{6ij}^{(6)} &= (\bar{\chi}\gamma_\mu\chi)\frac{1}{2}(
	\bar{u}\gamma_\mu u\delta_{iu}\delta_{ij}-\bar{u}\gamma_\mu\gamma_5u\delta_{iu}\delta_{ij}) + (\bar{\chi}\gamma_\mu\chi)\frac{1}{2}(V_{id}^*V_{jd}\bar{d}\gamma_\mu d - V_{id}^*V_{jd}\bar{d}\gamma_\mu\gamma_5d
	+ V_{is}^*V_{js}\bar{s}\gamma_\mu s - V_{is}^*V_{js}\bar{s}\gamma_\mu\gamma_5s) \\
	&= \frac{1}{2}(
	R_{2u}^{(6)}\delta_{iu}\delta_{ij}-R_{4u}^{(6)}\delta_{iu}\delta_{ij}) + \frac{1}{2}(V_{id}^*V_{jd}R_{2d}^{(6)} - V_{id}^*V_{jd}R_{4d}^{(6)}
	+ V_{is}^*V_{js}R_{2s}^{(6)} - V_{is}^*V_{js}R_{4s}^{(6)}) \\
	%%%%%%%%%%%%%%%%%%%%%%%%%%%%%%%%%%%%%%%%%%%%%%%%%%%%%%%%%%%%%%%%%%%%%%%%%%%%%%%%%%%%%%%%%%%%%%%%%%%%%%%%%%%%%%%%%%%%%%%
	Q_{7ij}^{(6)} &= (\bar{\chi}\gamma_\mu\gamma_5\chi)\frac{1}{2}(\bar{u}\gamma^\mu u\delta_{ij}\delta_{iu} + \bar{u}\gamma^\mu \gamma_5u\delta_{ij}\delta_{iu}) \\
	&= \frac{1}{2}(R_{2u}^{(6)}\delta_{ij}\delta_{iu} + R_{4u}^{(6)}\delta_{ij}\delta_{iu}) \\
	%%%%%%%%%%%%%%%%%%%%%%%%%%%%%%%%%%%%%%%%%%%%%%%%%%%%%%%%%%%%%%%%%%%%%%%%%%%%%%%%%%%%%%%%%%%%%%%%%%%%%%%%%%%%%%%%%%%%%%%
	Q_{8ij}^{(6)} &= (\bar{\chi}\gamma_\mu\gamma_5\chi)\frac{1}{2}(\bar{d}\gamma^\mu d\delta_{ij}\delta_{id} + \bar{d}\gamma^\mu\gamma_5d\delta_{ij}\delta_{id} + \bar{s}\gamma^\mu s\delta_{ij}\delta_{is} + \bar{s}\gamma^\mu \gamma_5s\delta_{ij}\delta_{is}) \\
	&= \frac{1}{2}(R_{2d}^{(6)}\delta_{ij}\delta_{id} + R_{4d}^{(6)}\delta_{ij}\delta_{id} + R_{2s}^{(6)}\delta_{ij}\delta_{is} + R_{4s}^{(6)}\delta_{ij}\delta_{is})
\end{align*}

\section{Ausrechnen der Koeffizienten}
\begin{align*}
	\mathcal{L}_Q^{(6)} = \sum_{k,i,j} C_{kij}Q_{kij}^{(6)} \overset{!}{=} \sum_{l,m}D_{lm}R_{lm}^{(6)} = \mathcal{L}_R^{(6)}
\end{align*}
\begin{align*}
	\mathcal{L}_Q^{(6)} &= \sum_{k,i,j} C_{kij}Q_{kij}^{(6)} \\
	&= \sum_{i,j} C_{1ij}(\frac{\delta_{3a}\tau_{0}}{4}(R_{1u}^{(6)}\delta_{ij}\delta_{iu}
	- V_{id}^*V_{jd}R_{1d}^{(6)} - V_{is}^*V_{js}R_{1s}^{(6)}) -\frac{\delta_{3a}\tau_{0}}{4} (R_{3u}^{(6)}\delta_{ij}\delta_{iu}
	- V_{id}^*V_{jd}R_{3d}^{(6)} - V_{is}^*V_{js}R_{3s}^{(6)})) \\
	&\quad+ C_{2ij}(\frac{1}{2}(
	R_{1u}^{(6)}\delta_{iu}\delta_{ij}-R_{3u}^{(6)}\delta_{iu}\delta_{ij}) + \frac{1}{2}(V_{id}^*V_{jd}R_{1d}^{(6)} - V_{id}^*V_{jd}R_{3d}^{(6)}
	+ V_{is}^*V_{js}R_{1s}^{(6)} - V_{is}^*V_{js}R_{3s}^{(6)})) \\
	&\quad+ C_{3ij}(\frac{1}{2}(R_{1u}^{(6)}\delta_{ij}\delta_{iu} + R_{3u}^{(6)}\delta_{ij}\delta_{iu})) \\
	&\quad+ C_{4ij}(\frac{1}{2}(R_{1d}^{(6)}\delta_{ij}\delta_{id} + R_{3d}^{(6)}\delta_{ij}\delta_{id} + R_{1s}^{(6)}\delta_{ij}\delta_{is} + R_{3s}^{(6)}\delta_{ij}\delta_{is})) \\
	&\quad+ C_{5ij}(\frac{\delta_{3a}\tau_{0}}{4}(R_{2u}^{(6)}\delta_{ij}\delta_{iu}
	- V_{id}^*V_{jd}R_{2d}^{(6)} - V_{is}^*V_{js}R_{2s}^{(6)}) -\frac{\delta_{3a}\tau_{0}}{4} (R_{4u}^{(6)}\delta_{ij}\delta_{iu}
	- V_{id}^*V_{jd}R_{4d}^{(6)} - V_{is}^*V_{js}R_{4s}^{(6)})) \\
	&\quad+ C_{6ij}(\frac{1}{2}(
	R_{2u}^{(6)}\delta_{iu}\delta_{ij}-R_{4u}^{(6)}\delta_{iu}\delta_{ij}) + \frac{1}{2}(V_{id}^*V_{jd}R_{2d}^{(6)} - V_{id}^*V_{jd}R_{4d}^{(6)}
	+ V_{is}^*V_{js}R_{2s}^{(6)} - V_{is}^*V_{js}R_{4s}^{(6)})) \\
	&\quad+ C_{7ij}(\frac{1}{2}(R_{2u}^{(6)}\delta_{ij}\delta_{iu} + R_{4u}^{(6)}\delta_{ij}\delta_{iu})) \\
	&\quad+ C_{8ij}(\frac{1}{2}(R_{2d}^{(6)}\delta_{ij}\delta_{id} + R_{4d}^{(6)}\delta_{ij}\delta_{id} + R_{2s}^{(6)}\delta_{ij}\delta_{is} + R_{4s}^{(6)}\delta_{ij}\delta_{is}))
\end{align*}
\begin{align*}
	&= \sum_{i,j} \\
	&+ R_{1u}^{(6)}\frac{\delta_{ij}\delta_{iu}}{2}\left(C_{1ij}\frac{\delta_{3a}\tau_{0}}{2}+ 
	C_{2ij} + C_{3ij}\right) \\
	&+ R_{1d}^{(6)}\frac{1}{2}\left(- V_{id}^*V_{jd}C_{1ij}\frac{\delta_{3a}\tau_{0}}{2}+ V_{id}^*V_{jd}C_{2ij}+ \delta_{ij}\delta_{id}C_{4ij}\right) \\
	&+ R_{1s}^{(6)}\frac{1}{2}\left(- V_{is}^*V_{js}C_{1ij}\frac{\delta_{3a}\tau_{0}}{2}+ V_{is}^*V_{js}C_{2ij}+ \delta_{ij}\delta_{is}C_{4ij}\right) \\
	%%%%%%%%%%%%%%%%%%%%%%%%%%%%%%%%%%
	&+ R_{2u}^{(6)}\frac{\delta_{ij}\delta_{iu}}{2}\left(\frac{\delta_{3a}\tau_{0}}{2}C_{5ij}+ 
	C_{6ij} + C_{7ij}\right) \\
	&+ R_{2d}^{(6)}\frac{1}{2}\left(- V_{id}^*V_{jd}\frac{\delta_{3a}\tau_{0}}{2}C_{5ij} + C_{6ij}V_{id}^*V_{jd} + \delta_{ij}\delta_{id}C_{8ij}\right) \\
	&+ R_{2s}^{(6)}\frac{1}{2}\left(- V_{is}^*V_{js}\frac{\delta_{3a}\tau_{0}}{2}C_{5ij} + C_{6ij}V_{is}^*V_{js} + \delta_{ij}\delta_{is}C_{8ij}\right) \\
	%%%%%%%%%%%%%%%%%%%%%%%%%%%%%%%%%%
	&+ R_{3u}^{(6)}\frac{\delta_{ij}\delta_{iu}}{2}\left(- C_{1ij}\frac{\delta_{3a}\tau_{0}}{2} - C_{2ij} + C_{3ij}\right) \\
	&+ R_{3d}^{(6)}\frac{1}{2}\left(C_{1ij}\frac{\delta_{3a}\tau_{0}}{2} V_{id}^*V_{jd} - V_{id}^*V_{jd}C_{2ij} + \delta_{ij}\delta_{id}C_{4ij}\right) \\
	&+ R_{3s}^{(6)}\frac{1}{2}\left(C_{1ij}\frac{\delta_{3a}\tau_{0}}{2} V_{is}^*V_{js} - V_{is}^*V_{js}C_{2ij} + \delta_{ij}\delta_{is}C_{4ij}\right) \\
	%%%%%%%%%%%%%%%%%%%%%%%%%%%%%%%%%%
	&+ R_{4u}^{(6)}\frac{\delta_{ij}\delta_{iu}}{2}\left(-\frac{\delta_{3a}\tau_{0}}{2}C_{5ij} - C_{6ij} + C_{7ij}\right) \\
	&+ R_{4d}^{(6)}\frac{1}{2}\left(\frac{\delta_{3a}\tau_{0}}{2}C_{5ij}V_{id}^*V_{jd} - C_{6ij}V_{id}^*V_{jd} + \delta_{ij}\delta_{id}C_{8ij}\right) \\
	&+ R_{4s}^{(6)}\frac{1}{2}\left(\frac{\delta_{3a}\tau_{0}}{2}C_{5ij}V_{is}^*V_{js} - C_{6ij}V_{is}^*V_{js} + \delta_{ij}\delta_{is}C_{8ij}\right) \\	
\end{align*}
\end{document}