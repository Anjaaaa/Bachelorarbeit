Operatoren aus Chiral Effective Theory heißen R,
\begin{align*}
	R_{1q}^{(6)} &= (\bar{\chi}\gamma_\mu\chi)(\bar{q}\gamma^\mu q) \\
	R_{2q}^{(6)} &= (\bar{\chi}\gamma_\mu\gamma_5\chi)(\bar{q}\gamma^\mu q) \\
	R_{3q}^{(6)} &= (\bar{\chi}\gamma_\mu\chi)(\bar{q}\gamma^\mu\gamma_5q) \\
	R_{4q}^{(6)} &= (\bar{\chi}\gamma_\mu\gamma_5\chi)(\bar{q}\gamma^\mu\gamma_5q)
\end{align*}
die anderen Q
\begin{align*}
	Q_{1ij}^{(6)} &= (\bar{\chi}\gamma_\mu\tilde{\tau}^a\chi)(\bar{Q}_L^i\gamma^\mu \tau^aQ_L^j) \\
	Q_{2ij}^{(6)} &= (\bar{\chi}\gamma_\mu\chi)(\bar{Q}_L^i\gamma^\mu Q_L^j) \\
	Q_{3ij}^{(6)} &= (\bar{\chi}\gamma_\mu\chi)(\bar{u}_R^i\gamma^\mu u_R^j) \\
	Q_{4ij}^{(6)} &= (\bar{\chi}\gamma_\mu\chi)(\bar{d}_R^i\gamma^\mu d_R^j) \\
	Q_{5ij}^{(6)} &= (\bar{\chi}\gamma_\mu\gamma_5\tilde{\tau}^a\chi)(\bar{Q}_L^i\gamma^\mu \tau^aQ_L^j) \\
	Q_{6ij}^{(6)} &= (\bar{\chi}\gamma_\mu\gamma_5\chi)(\bar{Q}_L^i\gamma^\mu Q_L^j) \\
	Q_{7ij}^{(6)} &= (\bar{\chi}\gamma_\mu\gamma_5\chi)(\bar{u}_R^i\gamma^\mu u_R^j) \\
	Q_{8ij}^{(6)} &= (\bar{\chi}\gamma_\mu\gamma_5\chi)(\bar{d}_R^i\gamma^\mu d_R^j)
\end{align*}
Das heißt es werden 12 (3 Flavour x 4 Operatoren) Koeffizienten auf 72 (8 Operatoren x 3x3 Möglichkeiten) Koeffizienten aufgeblasen, wobei sich die 72 auf 42 reduzieren durch die Annahmen \textit{nur u,d,s} und \textit{keine Mischterme}.