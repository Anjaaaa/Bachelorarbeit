\section{Cross Section}
We are only interested in the interactions $Q_{2bs}$, therefore all coefficients $C_{lij}$ vanish except $C_{2sb} = C_{2bs}^*$. In terms of the coefficients in \eqref{eq:CoeffNormal} this means
\begin{align}
	K_{1,d} &= +\text{Re}(V_{cd}^*V_{td}C_{2sb})\notag \\
	K_{1,s} &= +\text{Re}(V_{cs}^*V_{ts}C_{2sb})\notag \\
	K_{3,d} &= -\text{Re}(V_{cd}^*V_{td}C_{2sb})\notag \\
	K_{3,s} &= -\text{Re}(V_{cs}^*V_{ts}C_{2sb}) \ ,
\end{align}
and all the other $K_{l,q}$ are zero. \\
We are interested in spin-independent cross sections, therefore we only use the interactions $(\bar{\chi}\gamma^\mu\chi)(\bar{q}\gamma_\mu q)$. Protons do not contain strange quarks, so the lagrangian only consists of one interaction:
\begin{align}
	\mathcal{L} = K_{1,d}(\bar{\chi}\gamma^\mu\chi)(\bar{d}\gamma_\mu d) + \text{h.c.} \ .
\end{align}
Since this operator only counts the number of down quarks in the nucleus, the matrix element is
\begin{align}
	M = Z\cdot 2K_{1,d} + (A-Z)\cdot K_{1,d} \ .
\end{align}
\begin{align}
	\sigma_{0,\text{tree}}^\text{SI} &= \frac{\mu_{A\chi}^2}{A^2\pi}\left|ZC_p +(A-Z)C_n\right|^2 \ . \\
	&= \frac{\mu_{A\chi}^2}{A^2\pi}K_{1,d}^2\times\mathcal{O}(10^2)
\end{align}



In case of the spin-dependent cross section we have
\begin{align}
	\sigma_{0,\text{tree}}^\text{SD} &= \frac{\mu_{A\chi}^2}{A^2\pi}32\Lambda^2J(J+1)64G_F^2 \\
	&= \frac{\mu_{A\chi}^2}{A^2\pi}32\left(\frac{K_{3,d}}{J}\frac{\Delta d^{(n)}}{\sqrt{2}G_F}\frac{\mu}{3.826}\right)^2J(J+1)64G_F^2 \\
	&= \frac{\mu_{A\chi}^2}{A^2\pi}2^{10}K_{3,d}^2(\Delta d^{(n)})^2\left(\frac{\mu}{3.826}\right)^2\frac{J(J+1)}{J^2} \\
	&= \frac{\mu_{A\chi}^2}{A^2\pi}K_{3,d}^2\times\mathcal{O}(10^{3}10^{-2}10^{-2})
\end{align}

\begin{align}
	&\langle N|K_{1,d}\bar{d}\gamma^\mu d|N\rangle + \langle N|K_{1,u}\bar{u}\gamma^\mu u|N\rangle\notag \\
	= &Z\left(2K_{1,u}+K_{1,d}\right)(\bar{p}\gamma^\mu p) + (A-Z)\left(K_{1,u}+2K_{1,d}\right)(\bar{n}\gamma^\mu n)\notag \\
	\notag \\
	&\langle N|K_{3,d}\bar{d}\gamma^\mu\gamma_5 d|N\rangle + \langle N|K_{3,u}\bar{u}\gamma^\mu\gamma_5 u|N\rangle\notag \\
	= &Z(K_{3,d}2s^\mu\Delta d^{(p)} + K_{3,u}2s^\mu\Delta u^{(p)})(\bar{p}\gamma^\mu\gamma_5 p)\notag \\
	+ &(A-Z)(K_{3,d}2s^\mu\Delta d^{(n)} + K_{3,u}2s^\mu\Delta u^{(n)})(\bar{n}\gamma^\mu\gamma_5 n)\notag
\end{align}
\begin{figure}
	\centering
	\begin{tikzpicture}
\tikzstyle{centerArrow}=[decoration={
markings,
mark=at position 0.5 with {\fill (2pt,0)--(-2pt,2.31pt)--(-2pt,-2.31pt)--cycle;}}]
\begin{scope}[xshift=3cm,yshift=-2cm]
\def\xmove{2}
\def\ymove{1.25}
\def\centerShift{2}
\def\centerSize{0.08cm}
\coordinate (tCenter1) at (0,0);
\coordinate[fill, circle,inner sep=\centerSize] (tCenter2) at (0,-\centerShift cm);
\node (upperLeft) at (-\xmove,\ymove) {$\chi$};
\node (upperRight) at (\xmove,\ymove) {$\chi$};
\node (lowerLeft) at (-\xmove,-\centerShift cm-\ymove cm) {$b_L,s_L$};
\node (lowerRight) at (\xmove,-\centerShift cm-\ymove cm) {$s_L,b_L$};
\node at (0.5,-\centerShift/2) {$Z'$};
\draw [centerArrow,postaction={decorate}]  (upperLeft) -- (tCenter1) ;
\draw [centerArrow,postaction={decorate}]  (tCenter1) -- (upperRight) ;
\draw [centerArrow,postaction={decorate}]  (lowerLeft) -- (tCenter2) ;
\draw [centerArrow,postaction={decorate}]  (tCenter2) -- (lowerRight) ;
\draw [decoration={snake, segment length=1.5mm, amplitude=0.5mm},decorate] (tCenter1) -- (tCenter2) ;
\end{scope}
\end{tikzpicture}
	\caption{Direct detection tree level diagram}
	\label{fig:DD}
\end{figure}
\todo{Wieso ist eigentlich $Q_L^2\gamma_\mu Q_L^3 = s_L\gamma_\mu b_L$?}