In this chapter we include the flavour mixing into an existing formalism. We use the model described in \cite{ChiralEFT}. It provides a framework to calculate cross sections for the direct detection of dark matter. The model bases on a set of dimension-five, -six, and -seven operators. We restrict our calculations to the dimension-six operators, which are
\begin{align}\label{eq:normal}
	R_{1,q} &= (\bar{\chi}_0\gamma_\mu\chi_0)(\bar{q}\gamma^\mu q) \ , && &R_{3,q} &= (\bar{\chi}_0\gamma_\mu\chi_0)(\bar{q}\gamma^\mu\gamma_5q) \ , \notag \\
	R_{2,q} &= (\bar{\chi}_0\gamma_\mu\gamma_5\chi_0)(\bar{q}\gamma^\mu q) \ , &&	&R_{4,q} &= (\bar{\chi}_0\gamma_\mu\gamma_5\chi_0)(\bar{q}\gamma^\mu\gamma_5q) \ .
\end{align}
Here $q=(u,d,c,s,t,b)$ is a quark and $\chi_0$ is the component of the weak dark matter mutliplet that has no electric charge.


Since the CKM mixing only applies to the left-handed down-type quarks, we need to rewrite these operators in terms of the left- and righthanded particle functions to include the CKM matrix. These chiral operators are
\begin{align}\label{eq:chiral}
	Q_{1ij} &= (\bar{\chi}\gamma_\mu\tilde{\tau}^a\chi)(\bar{Q}_L^i\gamma^\mu \tau^aQ_L^j) \ , && &Q_{5ij} &= (\bar{\chi}\gamma_\mu\gamma_5\tilde{\tau}^a\chi)(\bar{Q}_L^i\gamma^\mu \tau^aQ_L^j) \ , \notag \\
	Q_{2ij} &= (\bar{\chi}\gamma_\mu\chi)(\bar{Q}_L^i\gamma^\mu Q_L^j) \ , && &Q_{6ij} &= (\bar{\chi}\gamma_\mu\gamma_5\chi)(\bar{Q}_L^i\gamma^\mu Q_L^j) \ , \notag \\
	Q_{3ij} &= (\bar{\chi}\gamma_\mu\chi)(\bar{U}_R^i\gamma^\mu U_R^j) \ , && &Q_{7ij} &= (\bar{\chi}\gamma_\mu\gamma_5\chi)(\bar{U}_R^i\gamma^\mu U_R^j) \ , \notag \\
	Q_{4ij} &= (\bar{\chi}\gamma_\mu\chi)(\bar{D}_R^i\gamma^\mu D_R^j) \ , && &Q_{8ij} &= (\bar{\chi}\gamma_\mu\gamma_5\chi)(\bar{D}_R^i\gamma^\mu D_R^j) \ .
\end{align}
The operators $\tilde{\tau}^a,\tau^a$ are the generators of the $SU(2)$ in the corresponding spin-representation. The left-handed quarks come in doublets, therefore the generators are the Pauli matrices $\sigma_a$:
\begin{align}
	\tau^a = \frac{\sigma_a}{2} \ . \notag
\end{align}
Regarding the dark matter, we do not know the size of the multiplet and thus use the general spin-representation
\begin{align}
	(\tilde{\tau}^1 \pm i\tilde{\tau}^2)_{\sigma'\sigma} &= \delta_{\sigma'\sigma\pm1}\sqrt{(j\mp\sigma)(j\pm\sigma+1)} \ , \notag \\
	\tilde{\tau}^3_{\sigma'\sigma} &= \sigma\delta_{\sigma'\sigma} \ , \notag
\end{align}
where $j$ is the spin value and $-j\leq\sigma,\sigma'\leq j$. However, we only keep the electrically uncharged component of the multiplet, $\chi_0$.
\todo{Warum behält man eigentlich nur die ungeladene Komponente?}

In the following sections, we go through the steps of including the CKM matrix in the formalism:
\begin{enumerate}
	\item Inclusion of the Flavour Mixing by replacing the pure left-handed down-type quarks with the mixed quarks.
	\item Rewriting the chiral particle functions in terms of the unchiral particle functions and projection operators.
	\item Writing down the entire interaction lagrangians in terms of the operators in \eqref{eq:normal} and \eqref{eq:chiral} separately, and then comparing the coefficients.
\end{enumerate}
\todo{Der letzte Punkt ist schlecht formuliert.}

\section{Including the Flavour Mixing}
The inclusion of the CKM matrix only affects the chiral operators with left-handed quarks: $Q_{1ij},Q_{2ij},Q_{5ij},Q_{6ij}$. Since the dark matter part of the interaction remains unchanged, we only look at the quark part of the interaction, which becomes
\begin{align*}
	\bar{Q}_L^i\gamma^\mu\tau^a Q_L^j &=  \begin{pmatrix}
	\bar{u}_L^i \\ \bar{d}_L^i
	\end{pmatrix}
	\gamma^\mu \begin{pmatrix}
	u_L^j \\ d_L^j
	\end{pmatrix}
	= \frac{1}{2}\bar{u}_L^i\gamma^\mu \sigma_a d_L^j + \frac{1}{2}\bar{d}_L^i\gamma^\mu\sigma_a d_L^j \\
	&= \frac{1}{2}\bar{u}_L^i\gamma^\mu \sigma_a u_L^j + \frac{1}{2}(V_{id}^*\bar{d}_L + V_{is}^*\bar{s}_L+V_{ib}^*\bar{b}_L)\gamma^\mu\sigma_a(V_{jd}d_L+V_{js}s_L+V_{jb}b_L)
\end{align*}
for $Q_{1ij}, Q_{5ij}$, respectively
\begin{align*}
	\bar{Q}_L^i\gamma^\mu Q_L^j &= \begin{pmatrix}
	\bar{u}_L^i \\ \bar{d}_L^i
	\end{pmatrix}
	\gamma^\mu \begin{pmatrix}
	u_L^j \\ d_L^j
	\end{pmatrix}
	= \bar{u}_L^i\gamma^\mu d_L^j + \bar{d}_L^i\gamma^\mu d_L^j \\
	&= \bar{u}_L^i\gamma^\mu u_L^j + (V_{id}^*\bar{d}_L + V_{is}^*\bar{s}_L+V_{ib}^*\bar{b}_L)\gamma^\mu(V_{jd}d_L+V_{js}s_L+V_{jb}b_L)
\end{align*}
for $Q_{2ij}, Q_{6ij}$. For simplicity we only keep the light quarks $u,d,s$ and neglect mixed terms. So the whole set of quark interactions is
\begin{align}
	\bar{Q}_L^i\gamma^\mu\tau^a Q_L^j &\approx \frac{1}{2}\bar{u}_L\gamma^\mu u_L\delta_{ij}\delta_{iu}\delta_{a3} - \frac{1}{2}\delta_{a3}(V_{id}^*V_{jd}\bar{d}_L\gamma^\mu d_L + V_{is}^*V_{js}\bar{s}_L\gamma^\mu s_L) \ , \notag \\
	\bar{Q}_L^i\gamma^\mu Q_L^j &\approx \bar{u}_L\gamma^\mu u_L\delta_{ij}\delta_{iu} + V_{id}^*V_{jd}\bar{d}_L\gamma^\mu d_L + V_{is}^*V_{js}\bar{s}_L\gamma^\mu s_L \ , \notag \\
	\bar{u}_R^i\gamma^\mu u_R^j &\approx\bar{u}_R\gamma^\mu u_R\delta_{ij}\delta_{iu} \ , \notag \\
	\bar{d}_R^i\gamma^\mu d_R^j &\approx \bar{d}_R\gamma^\mu d_R\delta_{ij}\delta_{id} + \bar{s}_R\gamma^\mu s_R\delta_{ij}\delta_{is} \ .
\end{align}
Note that, by only keeping diagonal interactions, we abolish the expressions including $\tau^1,\tau^2$, respectively $\tilde{\tau}^1,\tilde{\tau}^2$. Therefore the dark matter part of the interactions $Q_{1ij},Q_{5ij}$ becomes
\begin{align}
	\bar{\chi}\gamma_\mu\tilde{\tau}^3\chi &= \sigma^0\bar{\chi}_0\gamma_\mu\chi_0 \ , \\
	\bar{\chi}\gamma_\mu\gamma_5\tilde{\tau}^3\chi &= \sigma^0\bar{\chi}_0\gamma_\mu\gamma_5\chi_0 \ ,
\end{align}
because remember: we only want the electrically uncharged component of $\chi$.

\section{Replacing Chiral Particle Functions}
The next step is rewriting the chiral particle functions in terms of the unchiral particle functions using the projection operators $P_L,P_R$. This leads to
\begin{align}
	\bar{Q}_L^i\gamma^\mu\tau^a Q_L^j =&\frac{1}{4}(\bar{u}\gamma^\mu u\delta_{ij}\delta_{iu}\delta_{3a}
	- V_{id}^*V_{jd}\bar{d}\gamma^\mu d\delta_{3a} - V_{is}^*V_{js}\bar{s}\gamma^\mu s\delta_{3a})\notag \\
	-&\frac{1}{4} (\bar{u}\gamma^\mu \gamma_5u\delta_{ij}\delta_{iu}\delta_{3a} - V_{id}^*V_{jd}\bar{d}\gamma^\mu\gamma_5 d\delta_{3a} - V_{is}^*V_{js}\bar{s}\gamma^\mu\gamma_5 s\delta_{3a})\notag \\
	%%%%%%%%%%%%%%%%%%%%%%%%%%%%%%%%%%%%%%%%%%%%%%%%%%%%%%%%%%%%%%%%%%%%%%%%%%%%%%%%%%%%%%%%%%%%%%%%%%%%%%%%%%%%%%%%%%%%
	\bar{Q}_L^i\gamma^\mu Q_L^j =&\frac{1}{2}(
	\bar{u}\gamma^\mu u\delta_{iu}\delta_{ij} + V_{id}^*V_{jd}\bar{d}\gamma^\mu d
	+ V_{is}^*V_{js}\bar{s}\gamma^\mu s)\notag \\
	-&\frac{1}{2}(\bar{u}\gamma^\mu\gamma_5u\delta_{iu}\delta_{ij} + V_{id}^*V_{jd}\bar{d}\gamma^\mu\gamma_5d + V_{is}^*V_{js}\bar{s}\gamma^\mu\gamma_5s)\notag \\
	%%%%%%%%%%%%%%%%%%%%%%%%%%%%%%%%%%%%%%%%%%%%%%%%%%%%%%%%%%%%%%%%%%%%%%%%%%%%%%%%%%%%%%%%%%%%%%%%%%%%%%%%%%%%%%%%%%%%
	\bar{u}_R^i\gamma^\mu u_R^j =&\frac{1}{2}(\bar{u}\gamma^\mu u\delta_{ij}\delta_{iu} + \bar{u}\gamma^\mu \gamma_5u\delta_{ij}\delta_{iu})\notag \\
	%%%%%%%%%%%%%%%%%%%%%%%%%%%%%%%%%%%%%%%%%%%%%%%%%%%%%%%%%%%%%%%%%%%%%%%%%%%%%%%%%%%%%%%%%%%%%%%%%%%%%%%%%%%%%%%%%%%%%
	\bar{d}_R^i\gamma^\mu d_R^j =&\frac{1}{2}(\bar{d}\gamma^\mu d\delta_{ij}\delta_{id} + \bar{d}\gamma^\mu\gamma_5d\delta_{ij}\delta_{id} + \bar{s}\gamma^\mu s\delta_{ij}\delta_{is} + \bar{s}\gamma^\mu \gamma_5s\delta_{ij}\delta_{is}) \ .
\end{align}
At this point we can express the chiral operators in terms of the original operators from \eqref{eq:normal}:
\begin{align}\label{eq:dependenciesoperators}
	Q_{1ij} =&\frac{\delta_{3a}\sigma^0}{4}(R_{1u}\delta_{ij}\delta_{iu} - V_{id}^*V_{jd}R_{1d} - V_{is}^*V_{js}R_{1s})\notag \\
	-&\frac{\delta_{3a}\sigma^0}{4} (R_{3u}\delta_{ij}\delta_{iu} - V_{id}^*V_{jd}R_{3d} - V_{is}^*V_{js}R_{3s})\notag \\
	%%%%%%%%%%%%%%%%%%%%%%%%%%%%%%%%%%%%%%%%%%%%%%%%%%%%%%%%%%%%%%%%%%%%%%%%%%%%%%%%%%%%%%%%%%%%%%%%%%%%%%%%%%%%%%%%%%%%%%%%
	Q_{2ij} =&\frac{1}{2}(R_{1u}\delta_{iu}\delta_{ij}+ V_{id}^*V_{jd}R_{1d}+ V_{is}^*V_{js}R_{1s})\notag \\
	-&\frac{1}{2}(R_{3u}\delta_{iu}\delta_{ij} + V_{id}^*V_{jd}R_{3d} + V_{is}^*V_{js}R_{3s})\notag \\
	%%%%%%%%%%%%%%%%%%%%%%%%%%%%%%%%%%%%%%%%%%%%%%%%%%%%%%%%%%%%%%%%%%%%%%%%%%%%%%%%%%%%%%%%%%%%%%%%%%%%%%%%%%%%%%%%%%%%%%%
	Q_{3ij} =&\frac{1}{2}(R_{1u}\delta_{ij}\delta_{iu} + R_{3u}\delta_{ij}\delta_{iu})\notag \\
	%%%%%%%%%%%%%%%%%%%%%%%%%%%%%%%%%%%%%%%%%%%%%%%%%%%%%%%%%%%%%%%%%%%%%%%%%%%%%%%%%%%%%%%%%%%%%%%%%%%%%%%%%%%%%%%%%%%%%%%
	Q_{4ij} =&\frac{1}{2}(R_{1d}\delta_{ij}\delta_{id} + R_{3d}\delta_{ij}\delta_{id} + R_{1s}\delta_{ij}\delta_{is} + R_{3s}\delta_{ij}\delta_{is}) \ .
	%%%%%%%%%%%%%%%%%%%%%%%%%%%%%%%%%%%%%%%%%%%%%%%%%%%%%%%%%%%%%%%%%%%%%%%%%%%%%%%%%%%%%%%%%%%%%%%%%%%%%%%%%%%%%%%%%%%%%%%
%	Q_{5ij} &=\frac{\delta_{3a}\sigma^0}{4}(R_{2u}\delta_{ij}\delta_{iu}
%	- V_{id}^*V_{jd}R_{2d} - V_{is}^*V_{js}R_{2s})\notag \\
%	&-\frac{\delta_{3a}\sigma^0}{4} (R_{4u}\delta_{ij}\delta_{iu}
%	- V_{id}^*V_{jd}R_{4d} - V_{is}^*V_{js}R_{4s})\notag \\
%	%%%%%%%%%%%%%%%%%%%%%%%%%%%%%%%%%%%%%%%%%%%%%%%%%%%%%%%%%%%%%%%%%%%%%%%%%%%%%%%%%%%%%%%%%%%%%%%%%%%%%%%%%%%%%%%%%%%%%%%%
%	Q_{6ij} &= \frac{1}{2}(R_{2u}\delta_{iu}\delta_{ij} + V_{id}^*V_{jd}R_{2d} + V_{is}^*V_{js}R_{2s})\notag \\
%	&- \frac{1}{2}(R_{4u}\delta_{iu}\delta_{ij} + V_{id}^*V_{jd}R_{4d} + V_{is}^*V_{js}R_{4s})\notag \\
%	%%%%%%%%%%%%%%%%%%%%%%%%%%%%%%%%%%%%%%%%%%%%%%%%%%%%%%%%%%%%%%%%%%%%%%%%%%%%%%%%%%%%%%%%%%%%%%%%%%%%%%%%%%%%%%%%%%%%%%%
%	Q_{7ij} &= \frac{1}{2}(R_{2u}\delta_{ij}\delta_{iu} + R_{4u}\delta_{ij}\delta_{iu})\notag \\
%	%%%%%%%%%%%%%%%%%%%%%%%%%%%%%%%%%%%%%%%%%%%%%%%%%%%%%%%%%%%%%%%%%%%%%%%%%%%%%%%%%%%%%%%%%%%%%%%%%%%%%%%%%%%%%%%%%%%%%%%
%	Q_{8ij} &= \frac{1}{2}(R_{2d}\delta_{ij}\delta_{id} + R_{4d}\delta_{ij}\delta_{id} + R_{2s}\delta_{ij}\delta_{is} + R_{4s}\delta_{ij}\delta_{is}) \ .
\end{align}
The operators $Q_{5ij}-Q_{8ij}$ can be obtained from $Q_{1ij}-Q_{4ij}$ by replacing $R_{1q}\leftrightarrow R_{2q}$ and $R_{3q}\leftrightarrow R_{4q}$.


\section{Comparing Coefficients}
\todo{Dieser Text ist nicht schön.}
Our final step is expressing the coefficients $K_{l,q}$ of the original operators $R_{l,q}$ in \eqref{eq:normal} in terms of the coefficients $C_{lij}$ of the chiral operators $Q_{lij}$ in \eqref{eq:chiral}. To get there we look at the overall interaction. The interaction cannot depend on the choice of particle representation (chiral or non-chiral), therefore
\begin{align}
	\sum_{l,q} K_{l,q}R_{l,q} \overset{!}{=} \sum_{l,i,j}C_{lij}Q_{lij} \ .
\end{align}
By putting the interactions \eqref{eq:dependenciesoperators} into the right side of the equation and rearranging the expression in terms of the $R_{l,q}$, we conclude that $K_{l,q}$ must be equal to the terms in front of $R_{l,q}$ on the right side. We get the dependencies
\begin{align}\label{eq:CoeffNormal}
	K_{1,u} &= \sum_{i,j}\frac{\delta_{ij}\delta_{iu}}{2}\left(C_{1ij}\frac{\delta_{3a}\sigma^0}{2}+ 
	C_{2ij} + C_{3ij}\right)\notag \\
	K_{1,d} &= \sum_{i,j}\frac{1}{2}\left(-V_{id}^*V_{jd}C_{1ij}\frac{\delta_{3a}\sigma^0}{2}+ V_{id}^*V_{jd}C_{2ij} + \delta_{ij}\delta_{id}C_{4ij}\right)\notag \\
	K_{1,s} &= \sum_{i,j}\frac{1}{2}\left(- V_{is}^*V_{js}C_{1ij}\frac{\delta_{3a}\sigma^0}{2}+ V_{is}^*V_{js}C_{2ij}+ \delta_{ij}\delta_{is}C_{4ij}\right)\notag \\
	%%%%%%%%%%%%%%%%%%%%%%%%%%%%%%%%%%
	K_{2,u} &= \sum_{i,j}\frac{\delta_{ij}\delta_{iu}}{2}\left(\frac{\delta_{3a}\sigma^0}{2}C_{5ij}+ 
	C_{6ij} + C_{7ij}\right)\notag \\
	K_{2,d} &= \sum_{i,j}\frac{1}{2}\left(- V_{id}^*V_{jd}\frac{\delta_{3a}\sigma^0}{2}C_{5ij} + C_{6ij}V_{id}^*V_{jd} + \delta_{ij}\delta_{id}C_{8ij}\right)\notag \\
	K_{2,s} &= \sum_{i,j}\frac{1}{2}\left(- V_{is}^*V_{js}\frac{\delta_{3a}\sigma^0}{2}C_{5ij} + C_{6ij}V_{is}^*V_{js} + \delta_{ij}\delta_{is}C_{8ij}\right)\notag \\
	%%%%%%%%%%%%%%%%%%%%%%%%%%%%%%%%%%
	K_{3,u} &= \sum_{i,j}\frac{\delta_{ij}\delta_{iu}}{2}\left(- C_{1ij}\frac{\delta_{3a}\sigma^0}{2} - C_{2ij} + C_{3ij}\right)\notag \\
	K_{3,d} &= \sum_{i,j}\frac{1}{2}\left(C_{1ij}\frac{\delta_{3a}\sigma^0}{2} V_{id}^*V_{jd} - V_{id}^*V_{jd}C_{2ij} + \delta_{ij}\delta_{id}C_{4ij}\right)\notag \\
	K_{3,s} &= \sum_{i,j}\frac{1}{2}\left(C_{1ij}\frac{\delta_{3a}\sigma^0}{2} V_{is}^*V_{js} - V_{is}^*V_{js}C_{2ij} + \delta_{ij}\delta_{is}C_{4ij}\right)\notag %\\
	%%%%%%%%%%%%%%%%%%%%%%%%%%%%%%%%%%
\end{align}
\begin{align}
	K_{4,u} &= \sum_{i,j}\frac{\delta_{ij}\delta_{iu}}{2}\left(-\frac{\delta_{3a}\sigma^0}{2}C_{5ij} - C_{6ij} + C_{7ij}\right)\notag \\
	K_{4,d} &= \sum_{i,j}\frac{1}{2}\left(\frac{\delta_{3a}\sigma^0}{2}C_{5ij}V_{id}^*V_{jd} - C_{6ij}V_{id}^*V_{jd} + \delta_{ij}\delta_{id}C_{8ij}\right)\notag \\
	K_{4,s} &= \sum_{i,j}\frac{1}{2}\left(\frac{\delta_{3a}\sigma^0}{2}C_{5ij}V_{is}^*V_{js} - C_{6ij}V_{is}^*V_{js} + \delta_{ij}\delta_{is}C_{8ij}\right) \ .
\end{align}
To have a hermitian interaction the coefficients need to fulfil the relation $C_{lij} = C_{lji}^*$.