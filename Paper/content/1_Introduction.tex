We will take a look at the effects of the flavour mixing mechanism on the direct detection of dark matter. Therefore we expand an existing formalism for dark matter direct detection with the CKM matrix. Afterwards we present a new interaction proposed to explain anomalies in the decay $B\rightarrow Kl\bar{l}$ that also makes predictions about the direct detection of dark matter. Finally, we compare these predictions with the direct detection cross sections that respect flavour mixing.

Direct detection means detecting dark matter directly through interaction with a nucleus, in contrary to indirect detection, which means measuring secondary products of dark matter annihilation or dark matter decay.

\todo{Irgendwo muss noch kurz auf dunkle Materie eingegangen werden. Als Einleitung wären daher auch Detection Experimente gut. Evtl. hier direct vs indirect erklären.}


\begin{figure}
	\centering
	\begin{tikzpicture}
\tikzstyle{centerArrow}=[decoration={
    markings,
    mark=at position 0.5 with {\fill (2pt,0)--(-2pt,2.31pt)--(-2pt,-2.31pt)--cycle;}}]
\begin{scope}
\def\xmove{2.5}
\def\ymove{1.25}
\def\centerSize{0.15}
\node [fill, circle,inner sep=\centerSize cm] (tCenter)  {};
\node (upperLeft) at (-\xmove,\ymove) {$\chi$};
\node (upperRight) at (\xmove,\ymove) {$\chi$};
\node (lowerLeft) at (-\xmove,-\ymove) {$N$};
\node (lowerRight) at (\xmove,-\ymove) {$N$};
\draw [centerArrow,postaction={decorate}]  (upperLeft) -- (tCenter) ;
\draw [centerArrow,postaction={decorate}]  (lowerLeft) -- (tCenter) ;
\draw [centerArrow,postaction={decorate}]  (tCenter) -- (upperRight) ;
\draw [centerArrow,postaction={decorate}]  (tCenter) -- (lowerRight) ;
\end{scope}
\end{tikzpicture}
	\caption{Direct detection}
	\label{fig:DirectDetection}
\end{figure}