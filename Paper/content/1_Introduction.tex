Observing the movement of stars and planets is one of the earliest scientific acts of mankind, e. g. Babylonians analysed the orbit of the Venus around 1500 B.C. \cite{History}. Throughout all human civilisations night sky has fascinated people and is probably continuing to do so, because still little is known about the particles that make up the universe. Scientists only understand about 20\% of the universe's matter. They call the remaining 80\% dark matter, due to its lack of luminosity. \cite{DM}


Astronomers in the 1930's found first indications of its existence. One of them was F. Zwicky, a Swiss astronomer who was interested in the Coma cluster. Through measurements of the Doppler shifts in galactic spectra, he obtained the velocity distribution and the kinetic energy of the galaxies in the cluster. Zwicky assumed that there were only gravitational interactions that could be described with Newtonian gravity. The virial theorem then gives the relation $\langle T\rangle = -\sfrac{1}{2}\langle U\rangle$ between the average kinetic energy $\langle T\rangle$ and the average potential energy $\langle U\rangle$. Using this equation, Zwicky calculated the Cluster mass to be $M_\text{Coma} \approx \SI{4.5e-13}{}M_\text{sun}$. Surprisingly this was 50 times the mass obtained by luminosity measurements. Later, a small ratio of the missing mass could be accounted to intracluster gas, but a huge discrepancy remained and was attributed to dark matter. \cite{DM}


However, even today, after decades of intensive dark matter research, physicists do not know what dark matter particles are and how they interact. To shed light on dark matter, there currently are various detection experiments around the globe, which can be classified in two categories: indirect and direct detection. Indirect detection experiments try to measure the final products of dark matter annihilation whereas direct detection experiments aim at detecting the dark matter particles. As shown in figure \ref{fig:DirectDetection} this means scattering of a dark matter particle off a nucleus.


Nuclei consist of protons and neutrons, which contain quarks, hence quark flavour mixing might have an effect on direct detection. In this thesis we want to examine whether flavour mixing was rightfully neglected in one particular example. Therefore we first expand an existing formalism for dark matter direct detection with the CKM mixing matrix. Secondly, we present a new interaction proposed to explain anomalies in the decay $B\rightarrow Kl\bar{l}$ that also makes predictions about the direct detection of dark matter. And finally we compare these predictions with the direct detection cross sections that respect flavour mixing.


\begin{figure}
	\centering
	\begin{tikzpicture}
\tikzstyle{centerArrow}=[decoration={
    markings,
    mark=at position 0.5 with {\fill (2pt,0)--(-2pt,2.31pt)--(-2pt,-2.31pt)--cycle;}}]
\begin{scope}
\def\xmove{2.5}
\def\ymove{1.25}
\def\centerSize{0.15}
\node [fill, circle,inner sep=\centerSize cm] (tCenter)  {};
\node (upperLeft) at (-\xmove,\ymove) {$\chi$};
\node (upperRight) at (\xmove,\ymove) {$\chi$};
\node (lowerLeft) at (-\xmove,-\ymove) {$N$};
\node (lowerRight) at (\xmove,-\ymove) {$N$};
\draw [centerArrow,postaction={decorate}]  (upperLeft) -- (tCenter) ;
\draw [centerArrow,postaction={decorate}]  (lowerLeft) -- (tCenter) ;
\draw [centerArrow,postaction={decorate}]  (tCenter) -- (upperRight) ;
\draw [centerArrow,postaction={decorate}]  (tCenter) -- (lowerRight) ;
\end{scope}
\end{tikzpicture}
	\caption{Direct detection}
	\label{fig:DirectDetection}
\end{figure}