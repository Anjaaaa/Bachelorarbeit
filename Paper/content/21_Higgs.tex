We consider a complex scalar field $\Phi$ that interacts with itself through a potential
\begin{align}
	V(\Phi) = -\mu^2\Phi^\dagger\Phi + \frac{\lambda}{2}(\Phi^\dagger\Phi)^2 \ , \quad \mu^2 > 0 \ .
\end{align}
If $\lambda>0$, there are to minima of this potential. They occur at
\begin{align}
	\langle\Phi\rangle =  \pm\sqrt{\frac{\mu^2}{\lambda}} \ .
\end{align}
This is the vacuum expectation value of $\Phi$.

We add a $SU(2)$ gauge field coupled to $\Phi$, so $\Phi$ is a doublet $(\Phi_1,\Phi_2)$ with covariant derivative
\begin{align}
	D_\mu\Phi = (\partial_\mu - ig\sum_{a=1}^3A_\mu^a\tau^a)\Phi \ ,
\end{align}
where $\tau^a$ are the generators of $SU(2)$. In this case there is an infinite number of vacuum expection values for $\Phi$ arranged in a circle. We are free to choose one and make the simple choice
\begin{align}\label{eq:vev}
	\langle\Phi\rangle = \frac{1}{\sqrt{2}}\begin{pmatrix} 0 \\ v \end{pmatrix} \ , \qquad v = \sqrt{\frac{2\mu^2}{\lambda}} \ .
\end{align}
Note that by choosing one vacuum value, we break the symmetry.


The kinetic energy of $\Phi$ is
\begin{align}
	(D_\mu\Phi)^2 = &\frac{1}{2}(\partial_\mu v)(\partial^\mu v)\notag \\
	- &ig(\partial^\mu\begin{pmatrix} 0 & v \end{pmatrix}) \left(\sum_{a=1}^3A_\mu^a\tau^a\begin{pmatrix} 0 \\ v \end{pmatrix}\right)\notag \\
	- &\frac{1}{2}g^2\begin{pmatrix} 0 & v \end{pmatrix}\sum_{a,b=1}^{3}\tau^a\tau^b\begin{pmatrix} 0 \\ v \end{pmatrix}A_\mu^aA^{b\mu} \ .
\end{align}
Using the relation $\{\tau^a,\tau^b\}=\sfrac{1}{2}\cdot\delta_{ab}$, we can simplify the last expression to get
\begin{align}
	- \frac{1}{2}g^2\begin{pmatrix} 0 & v \end{pmatrix}\sum_{a,b=1}^{3}\tau^a\tau^b\begin{pmatrix} 0 \\ v \end{pmatrix}A_\mu^aA^{b\mu} &= - \frac{g^2v^2}{8}\sum_{a=1}^{3}A_\mu^aA^{a\mu} \ ,
\end{align}
which is a mass term $\mathcal{L}_m = -\frac{1}{2}m_A^2A_\mu A^\mu$ that assigns the mass $m_A = \frac{gv}{2}$ to all three gauge bosons. By expanding the system with an additional $U(1)$ symmetry, the kinetic energy would again provide three gauge boson masses, leaving the fourth gauge boson massless. The massive bosons can be identified as $W^\pm,Z^0$ and the massless as the photon.


The scalar field $\Phi$ is usually called Higgs boson. Obtaining particle mass terms in the kinetic energy of the Higgs, is, unsurprisingly, referred to as the Higgs mechanism. 
%
%
%We expand the system with an additional $U(1)$ symmetry with gauge boson $B$. The field $\Phi$ has a charge $Y_\Phi$ under $U(1)$. The new covariant derivative is
%\begin{align}
%	D_\mu\Phi = \left(\partial_\mu - ig\sum_{a=1}^3A_\mu^a\tau^2 - iY_\Phi g'B_\mu\right)\Phi \ .
%\end{align}
%Again, we examine the kinetic term
%\begin{align}
%	(D_\mu\Phi)^2 = &\frac{1}{2}(\partial_\mu v)(\partial^\mu v)\notag \\
%	- &\frac{1}{2}i\left(\partial^\mu\begin{pmatrix} 0 & v \end{pmatrix}\right) \left(g\sum_{a=1}^3A_\mu^a\tau^2 + Y_\Phi g'B_\mu\right)\begin{pmatrix} 0 \\ v \end{pmatrix}\notag \\
%	- &\frac{1}{2}\begin{pmatrix} 0 & v \end{pmatrix}
%	\left(g^2\sum_{a,b=1}^3A_\mu^aA^{b\mu}\tau^a\tau^b + 2gg'Y_\Phi\sum_{a=1}^3A_\mu^a\tau^aB^\mu + Y_\Phi^2g'^2B_\mu B^\mu\right)
%	\begin{pmatrix} 0 \\ v \end{pmatrix}
%\end{align}
%Using $\{\tau^a,\tau^b\}=\sfrac{1}{2}\cdot\delta_{ab}$ and replacing $\tau^a=\sfrac{\sigma^a}{2}$, we find for the last term
%\begin{align}
%	\mathcal{L}^{(\text{mass})} &= - \frac{v^2}{2}\left(g^2\frac{1}{4}\sum_{a=1}^3A_\mu^aA^{a\mu} + Y_\Phi^2g'^2B_\mu B^\mu - gg'Y_\Phi B^\mu A_\mu^3\right)\notag \\
%	&= - \frac{1}{2}\frac{v^2}{4}\left(g^2A_\mu^1A^{1\mu} + g^2A_\mu^2A^{2\mu} + (gA_\mu^3 - 2g'Y_\Phi B_\mu)^2\right)\notag \\
%	&= - \frac{1}{2}\frac{v^2}{4}\left(g^22W^+W^- + (g^2+4g'^2Y_\Phi^2)Z_0^2\right) \ .
%\end{align}
%Here we identified the known vector bosons
%\begin{align}
%	W_\mu^\pm &= \frac{1}{\sqrt{2}}(A_\mu^1\mp iA_\mu^2) \ , &&\quad &m_W &= \frac{v}{2}g \notag \\
%	Z_\mu^0 &= \frac{1}{\sqrt{g^2+4g'^2Y_\Phi^2}}(gA_\mu^3 - 2g'Y_\Phi B_\mu) \ , &&\quad &m_Z &= \frac{v}{2}\sqrt{g^2+4g'^2Y_\Phi^2}
%\end{align}
%as mass eigenstates of the gauge bosons.
%
%
%Since we are in a $SU(2)\times U(1)$ symmetry, there has to be a fourth gauge boson. As we have just derived, it is massless. We renounce giving an elaborate explanation for this, but we want to give a motivation. Therefore we need to remember that the masses arise through choosing a vacuum expectation value for $\Phi$ and thereby breaking the $SU(2)$ symmetry. Looking at the gauge transformation of $\Phi$
%\begin{align}
%	\Phi\rightarrow e^{i\sum_{a=1}^3\alpha^a\tau^a}e^{i\beta Y_\Phi}\Phi \ ,
%\end{align}
%we find that the choice $\alpha^1=\alpha^2=0$, $\alpha^3 = 2\beta Y_\Phi$ leaves the vacuum expectation value unchanged. Thus, parts of the symmetry are conserved and keep one gauge boson from acquiring mass. The fourth gauge boson is the photon, and it is orthogonal to $Z_\mu^0$:
%\begin{align}
%	A_\mu = \frac{1}{\sqrt{g^2 + 2g'^2Y_\Phi^2}}(2g'Y_\Phi A^3_\mu + gB_\mu) \ .
%\end{align}
