The origins of flavour mixing go back to the 1960s, when the Italian physicist Nicola Cabibbo resolved anomalies in data of weak interactions by proposing a flavour mixing of left-handed down-type quarks. Later in 1973, Kobayaski and Maskawa extended this idea to three quark generations to explain CP violation \cite{Griffiths}. On a mathematical level, quark flavour mixing arises from the fact that the fermion mass eigenstates do not necessarily coincide with the flavour eigenstates. In the course of this chapter, we first derive how the Higgs mechanism gives mass to particles and second take a look at fermion masses and why this leads to flavour mixing. The following calculations are based on the outlines in the textbook by Peskin and Schroeder \cite[Chapter 20]{Peskin} and the theoretical discussion of CP violation in \mbox{\cite[Chapter 1.2.1]{Tevatron}.}
\section{The Higgs Mechanism}
We consider a complex scalar field $\Phi$ that interacts with itself through a potential
\begin{align}
	V(\Phi) = -\mu^2\Phi^\dagger\Phi + \frac{\lambda}{2}(\Phi^\dagger\Phi)^2 \ , \quad \mu^2 > 0 \ .
\end{align}
If $\lambda>0$, there are two minima of this potential. They occur at
\begin{align}
	\langle\Phi\rangle =  \pm\sqrt{\frac{\mu^2}{\lambda}} \ .
\end{align}
This is the vacuum expectation value of $\Phi$.

We add a $SU(2)$ gauge field coupled to $\Phi$, so $\Phi$ is a doublet $(\Phi_1,\Phi_2)$ with covariant derivative
\begin{align}
	D_\mu\Phi = (\partial_\mu - ig\sum_{a=1}^3A_\mu^a\tau^a)\Phi \ ,
\end{align}
where $\tau^a$ are the generators of $SU(2)$. In this case there is an infinite number of vacuum expectation values for $\Phi$ arranged in a circle. We are free to choose one and make the simple choice
\begin{align}\label{eq:vev}
	\langle\Phi\rangle = \frac{1}{\sqrt{2}}\begin{pmatrix} 0 \\ v \end{pmatrix} \ , \qquad v = \sqrt{\frac{2\mu^2}{\lambda}} \ .
\end{align}
Note that by choosing one vacuum value, we break the symmetry.


The kinetic energy of $\Phi$ is
\begin{align}
	(D_\mu\Phi)^2 = &\frac{1}{2}(\partial_\mu v)(\partial^\mu v)\notag \\
	- &ig(\partial^\mu\begin{pmatrix} 0 & v \end{pmatrix}) \left(\sum_{a=1}^3A_\mu^a\tau^a\begin{pmatrix} 0 \\ v \end{pmatrix}\right)\notag \\
	- &\frac{1}{2}g^2\begin{pmatrix} 0 & v \end{pmatrix}\sum_{a,b=1}^{3}\tau^a\tau^b\begin{pmatrix} 0 \\ v \end{pmatrix}A_\mu^aA^{b\mu} \ .
\end{align}
Using the relation $\{\tau^a,\tau^b\}=\sfrac{1}{2}\cdot\delta_{ab}$, we can simplify the last expression to get
\begin{align}
	- \frac{1}{2}g^2\begin{pmatrix} 0 & v \end{pmatrix}\sum_{a,b=1}^{3}\tau^a\tau^b\begin{pmatrix} 0 \\ v \end{pmatrix}A_\mu^aA^{b\mu} &= - \frac{g^2v^2}{8}\sum_{a=1}^{3}A_\mu^aA^{a\mu} \ ,
\end{align}
which is a mass term $\mathcal{L}_m = -\frac{1}{2}m_A^2A_\mu A^\mu$ that assigns the mass $m_A = \frac{gv}{2}$ to all three gauge bosons. By expanding the system with an additional $U(1)$ symmetry, the kinetic energy would again provide three gauge boson masses, leaving the fourth gauge boson massless. The massive bosons can be identified as $W^\pm,Z^0$ and the massless as the photon.


The scalar field $\Phi$ is usually called Higgs boson. Obtaining particle mass terms in the kinetic energy of the Higgs, is, unsurprisingly, referred to as the Higgs mechanism. 
%
%
%We expand the system with an additional $U(1)$ symmetry with gauge boson $B$. The field $\Phi$ has a charge $Y_\Phi$ under $U(1)$. The new covariant derivative is
%\begin{align}
%	D_\mu\Phi = \left(\partial_\mu - ig\sum_{a=1}^3A_\mu^a\tau^2 - iY_\Phi g'B_\mu\right)\Phi \ .
%\end{align}
%Again, we examine the kinetic term
%\begin{align}
%	(D_\mu\Phi)^2 = &\frac{1}{2}(\partial_\mu v)(\partial^\mu v)\notag \\
%	- &\frac{1}{2}i\left(\partial^\mu\begin{pmatrix} 0 & v \end{pmatrix}\right) \left(g\sum_{a=1}^3A_\mu^a\tau^2 + Y_\Phi g'B_\mu\right)\begin{pmatrix} 0 \\ v \end{pmatrix}\notag \\
%	- &\frac{1}{2}\begin{pmatrix} 0 & v \end{pmatrix}
%	\left(g^2\sum_{a,b=1}^3A_\mu^aA^{b\mu}\tau^a\tau^b + 2gg'Y_\Phi\sum_{a=1}^3A_\mu^a\tau^aB^\mu + Y_\Phi^2g'^2B_\mu B^\mu\right)
%	\begin{pmatrix} 0 \\ v \end{pmatrix}
%\end{align}
%Using $\{\tau^a,\tau^b\}=\sfrac{1}{2}\cdot\delta_{ab}$ and replacing $\tau^a=\sfrac{\sigma^a}{2}$, we find for the last term
%\begin{align}
%	\mathcal{L}^{(\text{mass})} &= - \frac{v^2}{2}\left(g^2\frac{1}{4}\sum_{a=1}^3A_\mu^aA^{a\mu} + Y_\Phi^2g'^2B_\mu B^\mu - gg'Y_\Phi B^\mu A_\mu^3\right)\notag \\
%	&= - \frac{1}{2}\frac{v^2}{4}\left(g^2A_\mu^1A^{1\mu} + g^2A_\mu^2A^{2\mu} + (gA_\mu^3 - 2g'Y_\Phi B_\mu)^2\right)\notag \\
%	&= - \frac{1}{2}\frac{v^2}{4}\left(g^22W^+W^- + (g^2+4g'^2Y_\Phi^2)Z_0^2\right) \ .
%\end{align}
%Here we identified the known vector bosons
%\begin{align}
%	W_\mu^\pm &= \frac{1}{\sqrt{2}}(A_\mu^1\mp iA_\mu^2) \ , &&\quad &m_W &= \frac{v}{2}g \notag \\
%	Z_\mu^0 &= \frac{1}{\sqrt{g^2+4g'^2Y_\Phi^2}}(gA_\mu^3 - 2g'Y_\Phi B_\mu) \ , &&\quad &m_Z &= \frac{v}{2}\sqrt{g^2+4g'^2Y_\Phi^2}
%\end{align}
%as mass eigenstates of the gauge bosons.
%
%
%Since we are in a $SU(2)\times U(1)$ symmetry, there has to be a fourth gauge boson. As we have just derived, it is massless. We renounce giving an elaborate explanation for this, but we want to give a motivation. Therefore we need to remember that the masses arise through choosing a vacuum expectation value for $\Phi$ and thereby breaking the $SU(2)$ symmetry. Looking at the gauge transformation of $\Phi$
%\begin{align}
%	\Phi\rightarrow e^{i\sum_{a=1}^3\alpha^a\tau^a}e^{i\beta Y_\Phi}\Phi \ ,
%\end{align}
%we find that the choice $\alpha^1=\alpha^2=0$, $\alpha^3 = 2\beta Y_\Phi$ leaves the vacuum expectation value unchanged. Thus, parts of the symmetry are conserved and keep one gauge boson from acquiring mass. The fourth gauge boson is the photon, and it is orthogonal to $Z_\mu^0$:
%\begin{align}
%	A_\mu = \frac{1}{\sqrt{g^2 + 2g'^2Y_\Phi^2}}(2g'Y_\Phi A^3_\mu + gB_\mu) \ .
%\end{align}

\section{Fermion Masses and Flavour Mixing}
Hereafter, we describe how the standard model fermions get their masses and how this leads to quark flavour mixing. But beforehand, we introduce the notation used in this and the following chapters. The leptons and quark chiral particle multiplets are
\begin{align}
	E_R &= (e_R,\mu_R,\tau_R) \ , &&\quad &Y_E &= -2 \ ; \\
	L_L &= \left(\begin{pmatrix} \nu_e \\ e \end{pmatrix}_L,
	\begin{pmatrix} \nu_\mu \\ \mu \end{pmatrix}_L,
	\begin{pmatrix} \nu_\tau \\ \tau \end{pmatrix}_L\right) \ , &&\quad &Y_L &= -1 \ ; \\
	U_R &= (u_R,c_R,t_R) \ , &&\quad &Y_U &= \frac{4}{3} \ ; \\
	D_R &= (d_R,s_R,b_R) \ , &&\quad &Y_D &= -\frac{2}{3} \ ; \\
	Q_L &= \left(\begin{pmatrix} u \\ d \end{pmatrix}_L,
	\begin{pmatrix} c \\ s \end{pmatrix}_L,
	\begin{pmatrix} t \\ b \end{pmatrix}_L\right) \ , &&\quad &Y_Q &= \frac{1}{3} \ ;
\end{align}
where the hypercharge $Y$ is given. It is related to the electric charge $Q$ and the third component of the weak isospin $I_3$ through the Gell-Mann-Nishijima formula \cite[Chapter 10.7]{Griffiths}
\begin{align}\label{eq:GellMann}
	Q = \frac{Y}{2} + I_3 \ .
\end{align}
The right-handed particles are singlets under $SU(2)$ and therefore $I^{(r.h.)}_3=0$. We later also use the left-handed components $E_L = (e_L,\mu_L,\tau_L)$, $D_L = (d_L,s_L,b_L)$, and $U_L = (u_L,c_L,t_L)$.


The electroweak interaction lagrangian for the standard model fermions is
\begin{align}\label{eq:LFermion}
%	\mathcal{L}^{(\text{int})} = &\sum_{i=1}^{3}\bar{E}_R^i(i\slashed{\partial}-g_1Y_E\slashed{B})E_R^i + \bar{D}_R^i(i\slashed{\partial}-g_1Y_D\slashed{B})D_R^i + \bar{U}_R^i(i\slashed{\partial}-g_1Y_U\slashed{B})U_R^i \notag \\
%	+ &\sum_{i=1}^{3}\bar{L}_L^i(i\slashed{\partial}-g_1Y_L\slashed{B}-g_2\slashed{A})L_L^i + \bar{Q}_L^i(i\slashed{\partial}-g_1Y_Q\slashed{B}-g_2\slashed{A})Q_L^i \ ,
	\mathcal{L}^{(\text{int})} = &\bar{E}_R\gamma^\mu(i\partial_\mu-g_1Y_EB_\mu)E_R + \bar{L}_L\gamma^\mu(i\partial_\mu-g_1Y_LB_\mu-g_2\sum_{a=1}^{3}A_\mu^a\tau^a)L_L \notag \\
	+ &\bar{D}_R\gamma^\mu(i\partial_\mu-g_1Y_DB_\mu)D_R + \bar{U}_R\gamma^\mu(i\partial_\mu-g_1Y_UB_\mu)U_R \notag \\
	+ &\bar{Q}_L\gamma^\mu(i\partial_\mu-g_1Y_QB_\mu-g_2\sum_{a=1}^{3}A_\mu^a\tau^a)Q_L \ ,
\end{align}
where $B_\mu, A_\mu^a$ are the gauge bosons corresponding to $U(1)_Y\times SU(2)$. The coupling constants are $g_1$ and $g_2$, and the $\tau^a$ are again the $SU(2)$ generators. This lagrangian describes massless particles. In order to get a fermion mass term, one has to couple the left- and right-handed part of a particle. Since a direct coupling between a $SU(2)$ singlet and a $SU(2)$ doublet violates gauge invariance, a connecting field is necessary. To preserve invariance under Lorentz, $U(1)_Y$, and $SU(2)$ transformations, this field must have spin 0, hypercharge $Y=\sfrac{1}{2}$, and be a doublet. We identify this field with $\Phi$ from the previous chapter and write down the mass terms for the fermions
\begin{align}
%	\mathcal{L}^{(\text{mass})} = -\sum_{i,j=1}^{3}\left[\lambda^e_{ij} \bar{L}_L^i\Phi E_R^j  + \lambda_{ij}^d\bar{Q}_L^i\Phi D_R^j + \lambda_{ij}^u\sum_{a,b=1}^{2}\epsilon_{ab}\bar{Q}_{La}^{i}\Phi_b^\dagger U_R^j + \text{h.c.}\right] \ ,
	\mathcal{L}^{(\text{mass})} = -\left[\bar{L}_L\Phi\lambda^e  E_R  + \bar{Q}_L\Phi\lambda^d D_R + \bar{Q}_Li\sigma^2\Phi^\dagger\lambda^u U_R + \text{h.c.}\right] \ ,
\end{align}
with complex matrix coupling constants $\lambda^e,\lambda^d,\lambda^u$. Replacing $\Phi$ with its vacuum expectation value \eqref{eq:vev2} gives
\begin{align}\label{eq:LMass}
%	\mathcal{L}^{(\text{mass})} = -\frac{v}{\sqrt{2}}\sum_{i,j=1}^{3}\left[\lambda^e_{ij} \bar{E}_L^i E_R^j  + \lambda_{ij}^d\bar{D}_L^i D_R^j + \lambda_{ij}^u\bar{U}_L^{i} U_R^j + \text{h.c.}\right] \ ,
	\mathcal{L}^{(\text{mass})} = -\frac{v}{\sqrt{2}}\left[ \bar{E}_L\lambda^e E_R  + \bar{D}_L \lambda^dD_R + \bar{U}_L\lambda^u U_R + \text{h.c.}\right] \ .
\end{align}



The interaction lagrangian $\mathcal{L}^{(\text{int})}$ (see \eqref{eq:LFermion}) is invariant under the unitary transformations
\begin{align}
	E_L &\rightarrow S_eE_L \ , && &E_R &\rightarrow R_eE_R \ , \\
	U_L &\rightarrow S_uU_L \ , && &U_R &\rightarrow R_uU_R \ , \\
	D_L &\rightarrow S_dD_L \ , && &D_R &\rightarrow R_dD_R \ .
\end{align}
Thus, we can diagonalize the interactions in \eqref{eq:LMass}, using these transformations. The diagonal lepton coupling is $\tilde{\lambda}^e = S_e\lambda^eR_e^\dagger$ and parametrizes the lepton masses
\begin{align}
	m_e = \frac{v}{\sqrt{2}}\tilde{\lambda}_{11}^e \ , \quad m_\mu = \frac{v}{\sqrt{2}}\tilde{\lambda}_{22}^e \ , \quad m_\tau = \frac{v}{\sqrt{2}}\tilde{\lambda}_{33}^e \ .
\end{align}
The diagonal coupling for up-type quarks is $\tilde{\lambda}^u = S_u\lambda^uR_u^\dagger$, giving the corresponding masses
\begin{align}
	m_u = \frac{v}{\sqrt{2}}\tilde{\lambda}_{11}^u \ , \quad m_c = \frac{v}{\sqrt{2}}\tilde{\lambda}_{22}^u \ , \quad m_t = \frac{v}{\sqrt{2}}\tilde{\lambda}_{33}^u \ .
\end{align}
The transformed coupling of the down-type quarks is $\tilde{\lambda}^d = S_d\lambda^dR_d^\dagger$, leading to the down-type masses
\begin{align}
	m_d = \frac{v}{\sqrt{2}}\tilde{\lambda}_{11}^d \ , \quad m_s = \frac{v}{\sqrt{2}}\tilde{\lambda}_{22}^d \ , \quad m_b = \frac{v}{\sqrt{2}}\tilde{\lambda}_{33}^d \ .
\end{align}


The transformed particle multiplets are now mass eigenstates. But when looking at couplings of up- and down-type quarks, e. g. the current
\begin{align}
	\bar{U}_L\gamma^\mu D_L \ ,
\end{align}
the unitary transformations change the interaction to
\begin{align}
	\bar{U}_L\gamma^\mu S_u^\dagger S_d D_L \ ,
\end{align}
where we identify the CKM matrix $V=S_u^\dagger S_d$. Because $S_u,S_d$ are already determined by the diagonalizations above, the CKM matrix is not equal to the identity matrix in general. In fact, $V$ alters left-handed currents measurably. The most recent data by the Particle Data Group determines the parameters of the Wolfenstein parametrization
\begin{align}
	V = \begin{pmatrix}
	1-\frac{\lambda^2}{2} & \lambda & A\lambda^3(\rho-i\eta) \\
	-\lambda & 1-\frac{\lambda^2}{2} & A\lambda^2 \\
	A\lambda^3(1-\rho-i\eta) & -A\lambda^2 & 1
	\end{pmatrix}
\end{align}
to be (see \cite[Chapter 12]{PDG})
\begin{align}\label{eq:Wolf}
	\lambda &= \SI{0.22496(48)}{} \ , && &A &= \SI{0.823(13)}{} \ , \notag \\
	\rho &= \SI{0.141(19)}{} \ , && &\eta &= \SI{0.349(12)}{} \ .\footnotemark
\end{align}
\footnotetext{The Particle Data Group actually suggests two different sets of parameters obtained by two different methods. We choose one arbitrarily, because they only differ in the second decimal position, which does not make any difference in our calculations later.}