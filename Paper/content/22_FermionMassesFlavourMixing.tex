Before we discuss fermion mass terms, we introduce the notation we use in this and the following chapters. The leptons and quark chiral particle functions are
\begin{align}
	E_R &= (e_R,\mu_R,\tau_R) \ , &&\quad &Y_E &= -1 \ ; \\
	L_L &= \left(\begin{pmatrix} \nu_e \\ e \end{pmatrix}_L,
	\begin{pmatrix} \nu_\mu \\ \mu \end{pmatrix}_L,
	\begin{pmatrix} \nu_\tau \\ \tau \end{pmatrix}_L\right) \ , &&\quad &Y_L &= -1 \ ; \\
	U_R &= (u_R,c_R,t_R) \ , &&\quad &Y_U &= \frac{2}{3} \ ; \\
	D_R &= (d_R,s_R,b_R) \ , &&\quad &Y_D &= -\frac{1}{3} \ ; \\
	Q_L &= \left(\begin{pmatrix} u \\ d \end{pmatrix}_L,
	\begin{pmatrix} c \\ s \end{pmatrix}_L,
	\begin{pmatrix} t \\ b \end{pmatrix}_L\right) \ , &&\quad &Y_Q &= \frac{1}{3} \ ,
\end{align}
where the hypercharge $Y$ is given. It is related to the electric charge $Q$ and the third component of the weak isospin $I_3$ through the Gell-Mann-Nishijima formula $Y = 2(Q-I_3)$. The righthanded particles are singlets under $SU(2)$ and therefore $I^{(r.h.)}_3=0$. We also use the lefthanded components $E_L = (e_L,\mu_L,\tau_L)$, $D_L = (d_L,s_L,b_L)$, and $U_L = (u_L,c_L,t_L)$.

\todo{Die Ys nachschlagen.}


The electroweak interaction lagrangian for the standard model fermions is
\begin{align}\label{eq:LFermion}
%	\mathcal{L}^{(\text{int})} = &\sum_{i=1}^{3}\bar{E}_R^i(i\slashed{\partial}-g_1Y_E\slashed{B})E_R^i + \bar{D}_R^i(i\slashed{\partial}-g_1Y_D\slashed{B})D_R^i + \bar{U}_R^i(i\slashed{\partial}-g_1Y_U\slashed{B})U_R^i \notag \\
%	+ &\sum_{i=1}^{3}\bar{L}_L^i(i\slashed{\partial}-g_1Y_L\slashed{B}-g_2\slashed{A})L_L^i + \bar{Q}_L^i(i\slashed{\partial}-g_1Y_Q\slashed{B}-g_2\slashed{A})Q_L^i \ ,
	\mathcal{L}^{(\text{int})} = &\bar{E}_R\gamma^\mu(i\partial_\mu-g_1Y_EB_\mu)E_R + \bar{L}_L\gamma^\mu(i\partial_\mu-g_1Y_LB_\mu-g_2\sum_{i=1}^{3}A_\mu^i\tau^i)L_L \notag \\
	+ &\bar{D}_R\gamma^\mu(i\partial_\mu-g_1Y_DB_\mu)D_R + \bar{U}_R\gamma^\mu(i\partial_\mu-g_1Y_UB_\mu)U_R \notag \\
	+ &\bar{Q}_L\gamma^\mu(i\partial_\mu-g_1Y_QB_\mu-g_2\sum_{i=1}^{3}A_\mu^i\tau^i)Q_L \ ,
\end{align}
where $B_\mu, A_\mu^i$ are the gauge bosons corresponding to $U(1)_Y\times SU(2)$. The coupling constants are $g_1,g_2$, and $\tau^i$ are again the generators of $SU(2)$. This lagrangian describes massless particles. In order to get a fermion mass term, one has to couple the lefthanded and righthanded part of a particle. Since direct coupling between a $SU(2)$ singlet and a $SU(2)$ doublet violates gauge invariance, a connecting field is necessary. To preserve invariance under Lorentz, $U(1)_Y$, and $SU(2)$ transformations this field must have spin 0, hypercharge $Y=\sfrac{1}{2}$, and be a doublet. We identify this field with $\Phi$ from the previous chapter and write down the mass terms for the fermions
\begin{align}
%	\mathcal{L}^{(\text{mass})} = -\sum_{i,j=1}^{3}\left[\lambda^e_{ij} \bar{L}_L^i\Phi E_R^j  + \lambda_{ij}^d\bar{Q}_L^i\Phi D_R^j + \lambda_{ij}^u\sum_{a,b=1}^{2}\epsilon_{ab}\bar{Q}_{La}^{i}\Phi_b^\dagger U_R^j + \text{h.c.}\right] \ ,
	\mathcal{L}^{(\text{mass})} = -\left[\bar{L}_L\Phi\lambda^e  E_R  + \bar{Q}_L\Phi\lambda^d D_R + \bar{Q}_Li\sigma^2\Phi^\dagger\lambda^u U_R + \text{h.c.}\right] \ ,
\end{align}
with complex matrix coupling constants $\lambda^e,\lambda^d,\lambda^u$. Replacing $\Phi$ with its vacuum expectation value \eqref{eq:vev} gives
\begin{align}\label{eq:LMass}
%	\mathcal{L}^{(\text{mass})} = -\frac{v}{\sqrt{2}}\sum_{i,j=1}^{3}\left[\lambda^e_{ij} \bar{E}_L^i E_R^j  + \lambda_{ij}^d\bar{D}_L^i D_R^j + \lambda_{ij}^u\bar{U}_L^{i} U_R^j + \text{h.c.}\right] \ ,
	\mathcal{L}^{(\text{mass})} = -\frac{v}{\sqrt{2}}\left[ \bar{E}_L\lambda^e E_R  + \bar{D}_L \lambda^dD_R + \bar{U}_L\lambda^u U_R + \text{h.c.}\right] \ .
\end{align}



The interaction lagrangian \eqref{eq:LFermion} is invariant under unitary transformations
\begin{align}
	E_L &\rightarrow S_eE_L && &E_R &\rightarrow R_eE_R \ , \\
	U_L &\rightarrow S_uU_L && &U_R &\rightarrow R_uU_R \ , \\
	D_L &\rightarrow S_dD_L && &D_R &\rightarrow R_dD_R \ .
\end{align}
Thus, we can diagonalize the interactions in \eqref{eq:LMass}. The diagonal lepton coupling is $\tilde{\lambda}^e = S_e\lambda^eR_e^\dagger$ and parametrizes the lepton masses
\begin{align}
	m_e = \frac{v}{\sqrt{2}}\tilde{\lambda}_{11}^e \ , \quad m_\mu = \frac{v}{\sqrt{2}}\tilde{\lambda}_{22}^e \ , \quad m_\tau = \frac{v}{\sqrt{2}}\tilde{\lambda}_{33}^e \ .
\end{align}
The diagonal coupling for up-type quarks is $\tilde{\lambda}^u = S_u\lambda^uR_u^\dagger$, giving the corresponding masses
\begin{align}
	m_u = \frac{v}{\sqrt{2}}\tilde{\lambda}_{11}^u \ , \quad m_c = \frac{v}{\sqrt{2}}\tilde{\lambda}_{22}^u \ , \quad m_t = \frac{v}{\sqrt{2}}\tilde{\lambda}_{33}^u \ .
\end{align}
The transformed coupling of the down-type quarks is $\tilde{\lambda}^d = S_d\lambda^dR_d^\dagger$, leading to the down-type masses
\begin{align}
	m_d = \frac{v}{\sqrt{2}}\tilde{\lambda}_{11}^d \ , \quad m_s = \frac{v}{\sqrt{2}}\tilde{\lambda}_{22}^d \ , \quad m_b = \frac{v}{\sqrt{2}}\tilde{\lambda}_{33}^d \ .
\end{align}


The transformed particle functions are now mass eigenstates. But when looking at couplings of up- and down-type quarks, e.g. the current
\begin{align}
	\bar{U}_L\gamma^\mu D_L \ ,
\end{align}
the unitary transformations change the interaction to
\begin{align}
	\bar{U}_L\gamma^\mu S_u^\dagger S_d D_L \ ,
\end{align}
where we identify the CKM matrix $V = S_u^\dagger S_d$.