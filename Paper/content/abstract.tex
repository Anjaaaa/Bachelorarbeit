\thispagestyle{plain}
\section*{Abstract}
Quark flavour mixing is often neglected because of its small effects. But since dark matter only interacts weakly with baryonic matter, these tiny effects might be of importance for direct detection. This thesis examines whether flavour mixing was rightfully neglected by Altmannshofer et. al. in the discussion of a new interaction \cite{Z}. Therefore, we expand an existing framework for dark matter direct detection with the CKM mixing matrix. Thereafter, we present the new interaction, that was originally proposed to explain anomalies in the decay $B\rightarrow Kl\bar{l}$, but also makes predictions about the direct detection of dark matter. However, when treating dark matter cross sections, this new setup ignores flavour mixing effects. Thus, we compare those new cross sections with cross sections obtained with flavour mixing and conclude that, in this case, flavour mixing was rightfully neglected.


\section*{Kurzfassung}
Die Mischung der Quark-Flavours wird in Rechnungen häufig vernachlässigt. Allerdings sind die Wechselwirkungen dunkler Materie mit baryonischer Materie sehr schwach, sodass die Effekte der Flavour-Mischung hier bedeutsam sein könn\-ten. In dieser Arbeit wird untersucht, ob Altmannshofer et. al. die Flavour-Mischung bei der Betrachtung von direct detection unter einer neuen Wechselwirkung \cite{Z} richtigerweise nicht beachtet haben. Dazu wird zunächst ein bereits existierender Formalismus zur Berechnung von direct detection Wirkungsquerschnitten um die CKM-Mischungsmatrix erweitert. Danach wird die neue Wechselwirkung vorgestellt und der daraus resultierende Wirkungsquerschnitt für die direct detection dunkler Materie mit dem Wirkungsquerschnitt verglichen, den wird unter Berücksichtigung der Flavour-Mischung berechnen. Wir kommen zu dem Schluss, dass die Vernachlässigung der Flavour-Mischung in diesem Fall gerechtfertigt ist.