\thispagestyle{plain}
\section*{Abstract}
Quark flavour mixing often is neglected because of its little effects. But since dark matter only interacts weakly with baryonic matter, these tiny effects might be of importance for direct detection. This thesis examines the effects of flavour mixing on dark matter direct detection in one specific example. After explaining the theoretical origin of the flavour mixing, we therefore expand an existing mathematical framework for direct detection with the CKM mixing matrix. We then present a new interaction proposed to explain anomalies in the decay $B\rightarrow Kl\bar{l}$, which also makes predictions about the direct detection of dark matter. However, when treating dark matter cross sections, this new set-up ignores flavour mixing effects. Thus, we compare those new cross sections with cross sections obtained with flavour mixing and see that, in this case, flavour mixing was rightfully neglected.