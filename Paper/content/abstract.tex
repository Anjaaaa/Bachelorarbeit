\thispagestyle{plain}
\section*{Abstract}
Quark flavour mixing is often neglected because of its small effects. But since dark matter only interacts weakly with baryonic matter, these tiny effects might be of importance for direct detection. This thesis examines whether flavour mixing was rightfully neglected by Altmannshofer et al. in the discussion of a new \mbox{interaction \cite{Z}}. Therefore, we expand an existing framework for dark matter direct detection with the CKM mixing matrix. Thereafter, we present the new interaction, that was originally proposed to explain anomalies in the decay $B\rightarrow Kl\bar{l}$, but also makes predictions about the direct detection of dark matter. However, when treating dark matter cross sections, this new setup ignores flavour mixing effects. Thus, we compare those new cross sections with cross sections obtained with flavour mixing and conclude that, in this case, flavour mixing was rightfully neglected.


\section*{Kurzfassung}
Die Effekte der Mischung von Quark-Flavours werden häufig als vernachlässigbar betrachtet. Aller\-dings sind die Wechselwirkungen dunkler Materie mit baryonischer Materie sehr schwach, sodass die Effekte der Flavour-Mischung hier bedeutsam sein könn\-ten. In dieser Arbeit wird untersucht, ob Altmannshofer et al. die Flavour-Mischung bei der Streuung dunkler Materie am Atomkern durch eine neuen Wechselwirkung \cite{Z} berechtigterweise vernachlässigt haben. Dazu wird zunächst ein bereits existierender Formalismus zur Berechnung von Wirkungsquerschnitten der Streuung dunkler Materie am Atomkern um die CKM-Mischungsmatrix erweitert. Die daraus resultierenden Wirkungsquerschnitte werden dann mit den Wirkungsquerschnitten von Altmannshofer et al. verglichen. Dabei kommen wir zu dem Schluss, dass die Vernachlässigung der Flavour-Mischung in diesem Fall gerechtfertigt ist.