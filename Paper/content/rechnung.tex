In this chapter we include the flavour mixing into an existing formalism. We use the model described in \cite{ChiralEFT}. It provides a framework to calculate cross sections for the direct detection of dark matter. The model bases on a set of dimension-five, -six, and -seven operators. We restrict our calculations to the dimension-six operators, which are
\begin{align}\label{eq:normal}
	R_{1,q} &= (\bar{\chi}\gamma_\mu\chi)(\bar{q}\gamma^\mu q) && &R_{3,q} &= (\bar{\chi}\gamma_\mu\chi)(\bar{q}\gamma^\mu\gamma_5q)\notag \\
	R_{2,q} &= (\bar{\chi}\gamma_\mu\gamma_5\chi)(\bar{q}\gamma^\mu q) &&	&R_{4,q} &= (\bar{\chi}\gamma_\mu\gamma_5\chi)(\bar{q}\gamma^\mu\gamma_5q) \ .
\end{align}
Since the CKM mixing only applies to the lefthanded down-type quarks, we need to rewrite these operators in terms of the left- and righthanded particle functions to include the CKM matrix. These chiral operators are
\begin{align}\label{eq:chiral}
	Q_{1ij} &= (\bar{\chi}\gamma_\mu\tilde{\tau}^a\chi)(\bar{Q}_L^i\gamma^\mu \tau^aQ_L^j) && &Q_{5ij} &= (\bar{\chi}\gamma_\mu\gamma_5\tilde{\tau}^a\chi)(\bar{Q}_L^i\gamma^\mu \tau^aQ_L^j)\notag \\
	Q_{2ij} &= (\bar{\chi}\gamma_\mu\chi)(\bar{Q}_L^i\gamma^\mu Q_L^j) && &Q_{6ij} &= (\bar{\chi}\gamma_\mu\gamma_5\chi)(\bar{Q}_L^i\gamma^\mu Q_L^j)\notag \\
	Q_{3ij} &= (\bar{\chi}\gamma_\mu\chi)(\bar{u}_R^i\gamma^\mu u_R^j) && &Q_{7ij} &= (\bar{\chi}\gamma_\mu\gamma_5\chi)(\bar{u}_R^i\gamma^\mu u_R^j)\notag \\
	Q_{4ij} &= (\bar{\chi}\gamma_\mu\chi)(\bar{d}_R^i\gamma^\mu d_R^j) && &Q_{8ij} &= (\bar{\chi}\gamma_\mu\gamma_5\chi)(\bar{d}_R^i\gamma^\mu d_R^j) \ ,
\end{align}
here $Q_L^i = (u_L^i, d_L^i)$ is the isospin doublet of the $i^\text{th}$ quark generation. The operators $\tilde{\tau}^a,\tau^a$ are the generators of the SU(2) in the corresponding spin-representation. So for the quarks, we use
\begin{align}
	\tau^a = \frac{\sigma_a}{2} \ ,
\end{align}
where $\sigma_a$ are the pauli matrices. Regarding the dark matter particle, we decide to only keep the terms with $\tilde{\tau}^a = \tilde{\tau}^3$ and get
\begin{align}
	\tilde{\tau}^3\chi = \tau_0\chi \ ,
\end{align}
with the weak isospin value $\tau_0$ of the dark matter particle.

Including the CKM matrix in the fromalism follows these steps:
\begin{enumerate}
	\item Replace the pure lefthanded down-type quarks with the mixed quarks.
	\item Rewrite the chiral particle functions in terms of the normal particle functions and projection operators.
	\item Write down the entire interaction lagrangian in terms of \eqref{eq:normal} and \eqref{eq:chiral} and compare the coefficients.
\end{enumerate}

\section{Including the Flavour Mixing}
The inclusion of the CKM matrix only affects the chiral operators $Q_{1ij},Q_{2ij},Q_{5ij},Q_{6ij}$. The quark part of the interaction with flavour mixing is therefore
\begin{align*}
	\bar{Q}_L^i\gamma^\mu Q_L^j &= \begin{pmatrix}
	\bar{u}_L^i \\ \bar{d}_L^i
	\end{pmatrix}
	\gamma^\mu \begin{pmatrix}
	u_L^j \\ d_L^j
	\end{pmatrix}
	= \bar{u}_L^i\gamma^\mu d_L^j + \bar{d}_L^i\gamma^\mu d_L^j \\
	&= \bar{u}_L^i\gamma^\mu u_L^j + (V_{id}^*\bar{d}_L + V_{is}^*\bar{s}_L+V_{ib}^*\bar{b}_L)\gamma^\mu(V_{jd}d_L+V_{js}s_L+V_{jb}b_L)
\end{align*}
for $Q_{2ij}, Q_{6ij}$, respectively
\begin{align*}
	\bar{Q}_L^i\gamma^\mu\tau^a Q_L^j &=  \begin{pmatrix}
	\bar{u}_L^i \\ \bar{d}_L^i
	\end{pmatrix}
	\gamma^\mu \begin{pmatrix}
	u_L^j \\ d_L^j
	\end{pmatrix}
	= \frac{1}{2}\bar{u}_L^i\gamma^\mu \sigma_a d_L^j + \frac{1}{2}\bar{d}_L^i\gamma^\mu\sigma_a d_L^j \\
	&= \frac{1}{2}\bar{u}_L^i\gamma^\mu \sigma_a u_L^j + \frac{1}{2}(V_{id}^*\bar{d}_L + V_{is}^*\bar{s}_L+V_{ib}^*\bar{b}_L)\gamma^\mu\sigma_a(V_{jd}d_L+V_{js}s_L+V_{jb}b_L)
\end{align*}
for $Q_{1ij}, Q_{5ij}$. For simplicity we only keep the light quarks $u,d,s$ and neglect mixed terms. So the whole set of quark interactions is
\begin{align}
	\bar{Q}_L^i\gamma^\mu\tau^a Q_L^j &\approx \frac{1}{2}\bar{u}_L\gamma^\mu u_L\delta_{ij}\delta_{iu}\delta_{a3} - \frac{1}{2}\delta_{a3}(V_{id}^*V_{jd}\bar{d}_L\gamma^\mu d_L + V_{is}^*V_{js}\bar{s}_L\gamma^\mu s_L)\notag \\
	\bar{Q}_L^i\gamma^\mu Q_L^j &\approx \bar{u}_L\gamma^\mu u_L\delta_{ij}\delta_{iu} + V_{id}^*V_{jd}\bar{d}_L\gamma^\mu d_L + V_{is}^*V_{js}\bar{s}_L\gamma^\mu s_L\notag \\
	\bar{u}_R^i\gamma^\mu u_R^j &\approx\bar{u}_R\gamma^\mu u_R\delta_{ij}\delta_{iu}\notag \\
	\bar{d}_R^i\gamma^\mu d_R^j &\approx \bar{d}_R\gamma^\mu d_R\delta_{ij}\delta_{id} + \bar{s}_R\gamma^\mu s_R\delta_{ij}\delta_{is} \ .
\end{align}

\section{Replacing Chiral Particle Functions}
The next step is rewriting the chiral particle functions in terms of the normal particle functions using the projection operators $P_L,P_R$. This leads us to
\begin{align}
	\bar{Q}_L^i\gamma^\mu\tau^a Q_L^j =&\frac{1}{4}(\bar{u}\gamma^\mu u\delta_{ij}\delta_{iu}\delta_{3a}
	- V_{id}^*V_{jd}\bar{d}\gamma^\mu d\delta_{3a} - V_{is}^*V_{js}\bar{s}\gamma^\mu s\delta_{3a})\notag \\
	-&\frac{1}{4} (\bar{u}\gamma^\mu \gamma_5u\delta_{ij}\delta_{iu}\delta_{3a} - V_{id}^*V_{jd}\bar{d}\gamma^\mu\gamma_5 d\delta_{3a} - V_{is}^*V_{js}\bar{s}\gamma^\mu\gamma_5 s\delta_{3a})\notag \\
	%%%%%%%%%%%%%%%%%%%%%%%%%%%%%%%%%%%%%%%%%%%%%%%%%%%%%%%%%%%%%%%%%%%%%%%%%%%%%%%%%%%%%%%%%%%%%%%%%%%%%%%%%%%%%%%%%%%%
	\bar{Q}_L^i\gamma^\mu Q_L^j =&\frac{1}{2}(
	\bar{u}\gamma^\mu u\delta_{iu}\delta_{ij} + V_{id}^*V_{jd}\bar{d}\gamma^\mu d
	+ V_{is}^*V_{js}\bar{s}\gamma^\mu s)\notag \\
	-&\frac{1}{2}(\bar{u}\gamma^\mu\gamma_5u\delta_{iu}\delta_{ij} + V_{id}^*V_{jd}\bar{d}\gamma^\mu\gamma_5d + V_{is}^*V_{js}\bar{s}\gamma^\mu\gamma_5s)\notag \\
	%%%%%%%%%%%%%%%%%%%%%%%%%%%%%%%%%%%%%%%%%%%%%%%%%%%%%%%%%%%%%%%%%%%%%%%%%%%%%%%%%%%%%%%%%%%%%%%%%%%%%%%%%%%%%%%%%%%%
	\bar{u}_R^i\gamma^\mu u_R^j =&\frac{1}{2}(\bar{u}\gamma^\mu u\delta_{ij}\delta_{iu} + \bar{u}\gamma^\mu \gamma_5u\delta_{ij}\delta_{iu})\notag \\
	%%%%%%%%%%%%%%%%%%%%%%%%%%%%%%%%%%%%%%%%%%%%%%%%%%%%%%%%%%%%%%%%%%%%%%%%%%%%%%%%%%%%%%%%%%%%%%%%%%%%%%%%%%%%%%%%%%%%%
	\bar{d}_R^i\gamma^\mu d_R^j =&\frac{1}{2}(\bar{d}\gamma^\mu d\delta_{ij}\delta_{id} + \bar{d}\gamma^\mu\gamma_5d\delta_{ij}\delta_{id} + \bar{s}\gamma^\mu s\delta_{ij}\delta_{is} + \bar{s}\gamma^\mu \gamma_5s\delta_{ij}\delta_{is}) \ .
\end{align}
At this point we can express the chiral operators in terms of the original operators from \eqref{eq:normal}:
\begin{align}\label{eq:dependenciesoperators}
	Q_{1ij} =&\frac{\delta_{3a}\tau_{0}}{4}(R_{1u}\delta_{ij}\delta_{iu} - V_{id}^*V_{jd}R_{1d} - V_{is}^*V_{js}R_{1s})\notag \\
	-&\frac{\delta_{3a}\tau_{0}}{4} (R_{3u}\delta_{ij}\delta_{iu} - V_{id}^*V_{jd}R_{3d} - V_{is}^*V_{js}R_{3s})\notag \\
	%%%%%%%%%%%%%%%%%%%%%%%%%%%%%%%%%%%%%%%%%%%%%%%%%%%%%%%%%%%%%%%%%%%%%%%%%%%%%%%%%%%%%%%%%%%%%%%%%%%%%%%%%%%%%%%%%%%%%%%%
	Q_{2ij} =&\frac{1}{2}(R_{1u}\delta_{iu}\delta_{ij}+ V_{id}^*V_{jd}R_{1d}+ V_{is}^*V_{js}R_{1s})\notag \\
	-&\frac{1}{2}(R_{3u}\delta_{iu}\delta_{ij} + V_{id}^*V_{jd}R_{3d} + V_{is}^*V_{js}R_{3s})\notag \\
	%%%%%%%%%%%%%%%%%%%%%%%%%%%%%%%%%%%%%%%%%%%%%%%%%%%%%%%%%%%%%%%%%%%%%%%%%%%%%%%%%%%%%%%%%%%%%%%%%%%%%%%%%%%%%%%%%%%%%%%
	Q_{3ij} =&\frac{1}{2}(R_{1u}\delta_{ij}\delta_{iu} + R_{3u}\delta_{ij}\delta_{iu})\notag \\
	%%%%%%%%%%%%%%%%%%%%%%%%%%%%%%%%%%%%%%%%%%%%%%%%%%%%%%%%%%%%%%%%%%%%%%%%%%%%%%%%%%%%%%%%%%%%%%%%%%%%%%%%%%%%%%%%%%%%%%%
	Q_{4ij} =&\frac{1}{2}(R_{1d}\delta_{ij}\delta_{id} + R_{3d}\delta_{ij}\delta_{id} + R_{1s}\delta_{ij}\delta_{is} + R_{3s}\delta_{ij}\delta_{is}) \ .
	%%%%%%%%%%%%%%%%%%%%%%%%%%%%%%%%%%%%%%%%%%%%%%%%%%%%%%%%%%%%%%%%%%%%%%%%%%%%%%%%%%%%%%%%%%%%%%%%%%%%%%%%%%%%%%%%%%%%%%%
%	Q_{5ij} &=\frac{\delta_{3a}\tau_{0}}{4}(R_{2u}\delta_{ij}\delta_{iu}
%	- V_{id}^*V_{jd}R_{2d} - V_{is}^*V_{js}R_{2s})\notag \\
%	&-\frac{\delta_{3a}\tau_{0}}{4} (R_{4u}\delta_{ij}\delta_{iu}
%	- V_{id}^*V_{jd}R_{4d} - V_{is}^*V_{js}R_{4s})\notag \\
%	%%%%%%%%%%%%%%%%%%%%%%%%%%%%%%%%%%%%%%%%%%%%%%%%%%%%%%%%%%%%%%%%%%%%%%%%%%%%%%%%%%%%%%%%%%%%%%%%%%%%%%%%%%%%%%%%%%%%%%%%
%	Q_{6ij} &= \frac{1}{2}(R_{2u}\delta_{iu}\delta_{ij} + V_{id}^*V_{jd}R_{2d} + V_{is}^*V_{js}R_{2s})\notag \\
%	&- \frac{1}{2}(R_{4u}\delta_{iu}\delta_{ij} + V_{id}^*V_{jd}R_{4d} + V_{is}^*V_{js}R_{4s})\notag \\
%	%%%%%%%%%%%%%%%%%%%%%%%%%%%%%%%%%%%%%%%%%%%%%%%%%%%%%%%%%%%%%%%%%%%%%%%%%%%%%%%%%%%%%%%%%%%%%%%%%%%%%%%%%%%%%%%%%%%%%%%
%	Q_{7ij} &= \frac{1}{2}(R_{2u}\delta_{ij}\delta_{iu} + R_{4u}\delta_{ij}\delta_{iu})\notag \\
%	%%%%%%%%%%%%%%%%%%%%%%%%%%%%%%%%%%%%%%%%%%%%%%%%%%%%%%%%%%%%%%%%%%%%%%%%%%%%%%%%%%%%%%%%%%%%%%%%%%%%%%%%%%%%%%%%%%%%%%%
%	Q_{8ij} &= \frac{1}{2}(R_{2d}\delta_{ij}\delta_{id} + R_{4d}\delta_{ij}\delta_{id} + R_{2s}\delta_{ij}\delta_{is} + R_{4s}\delta_{ij}\delta_{is}) \ .
\end{align}
The operators $Q_{5ij}-Q_{8ij}$ can be obtained from $Q_{1ij}-Q_{4ij}$ by replacing $R_{1q}\leftrightarrow R_{2q}$ and $R_{3q}\leftrightarrow R_{4q}$.

\section{Comparing Coefficients}
Our final step is expressing the coefficients $K_{l,q}$ of the original operators $R_{l,q}$ in \eqref{eq:normal} in terms of the coefficients $C_{lij}$ of the chiral operators $Q_{lij}$ in \eqref{eq:chiral}. To get there we look at the overall interaction. The interaction cannot depend on the representation of particle functions we choose, either normal or chiral, therefore
\begin{align}
	\sum_{l,q} K_{l,q}R_{l,q} \overset{!}{=} \sum_{l,i,j}C_{lij}Q_{lij} \ .
\end{align}
By putting the interactions \eqref{eq:dependenciesoperators} into the right side of the equation and rearranging the expression in terms of the $R_{l,q}$, we conclude that $K_{l,q}$ must be equal to the terms in front of $R_{l,q}$ on the right side. We get the dependencies
\begin{align}\label{eq:CoeffNormal}
	K_{1,u} &= \sum_{i,j}\frac{\delta_{ij}\delta_{iu}}{2}\left(C_{1ij}\frac{\delta_{3a}\tau_{0}}{2}+ 
	C_{2ij} + C_{3ij}\right)\notag \\
	K_{1,d} &= \sum_{i,j}\frac{1}{2}\left(-V_{id}^*V_{jd}C_{1ij}\frac{\delta_{3a}\tau_{0}}{2}+ V_{id}^*V_{jd}C_{2ij} + \delta_{ij}\delta_{id}C_{4ij}\right)\notag \\
	K_{1,s} &= \sum_{i,j}\frac{1}{2}\left(- V_{is}^*V_{js}C_{1ij}\frac{\delta_{3a}\tau_{0}}{2}+ V_{is}^*V_{js}C_{2ij}+ \delta_{ij}\delta_{is}C_{4ij}\right)\notag \\
	%%%%%%%%%%%%%%%%%%%%%%%%%%%%%%%%%%
	K_{2,u} &= \sum_{i,j}\frac{\delta_{ij}\delta_{iu}}{2}\left(\frac{\delta_{3a}\tau_{0}}{2}C_{5ij}+ 
	C_{6ij} + C_{7ij}\right)\notag \\
	K_{2,d} &= \sum_{i,j}\frac{1}{2}\left(- V_{id}^*V_{jd}\frac{\delta_{3a}\tau_{0}}{2}C_{5ij} + C_{6ij}V_{id}^*V_{jd} + \delta_{ij}\delta_{id}C_{8ij}\right)\notag \\
	K_{2,s} &= \sum_{i,j}\frac{1}{2}\left(- V_{is}^*V_{js}\frac{\delta_{3a}\tau_{0}}{2}C_{5ij} + C_{6ij}V_{is}^*V_{js} + \delta_{ij}\delta_{is}C_{8ij}\right)\notag \\
	%%%%%%%%%%%%%%%%%%%%%%%%%%%%%%%%%%
	K_{3,u} &= \sum_{i,j}\frac{\delta_{ij}\delta_{iu}}{2}\left(- C_{1ij}\frac{\delta_{3a}\tau_{0}}{2} - C_{2ij} + C_{3ij}\right)\notag \\
	K_{3,d} &= \sum_{i,j}\frac{1}{2}\left(C_{1ij}\frac{\delta_{3a}\tau_{0}}{2} V_{id}^*V_{jd} - V_{id}^*V_{jd}C_{2ij} + \delta_{ij}\delta_{id}C_{4ij}\right)\notag \\
	K_{3,s} &= \sum_{i,j}\frac{1}{2}\left(C_{1ij}\frac{\delta_{3a}\tau_{0}}{2} V_{is}^*V_{js} - V_{is}^*V_{js}C_{2ij} + \delta_{ij}\delta_{is}C_{4ij}\right)\notag \\
	%%%%%%%%%%%%%%%%%%%%%%%%%%%%%%%%%%
	K_{4,u} &= \sum_{i,j}\frac{\delta_{ij}\delta_{iu}}{2}\left(-\frac{\delta_{3a}\tau_{0}}{2}C_{5ij} - C_{6ij} + C_{7ij}\right)\notag \\
	K_{4,d} &= \sum_{i,j}\frac{1}{2}\left(\frac{\delta_{3a}\tau_{0}}{2}C_{5ij}V_{id}^*V_{jd} - C_{6ij}V_{id}^*V_{jd} + \delta_{ij}\delta_{id}C_{8ij}\right)\notag \\
	K_{4,s} &= \sum_{i,j}\frac{1}{2}\left(\frac{\delta_{3a}\tau_{0}}{2}C_{5ij}V_{is}^*V_{js} - C_{6ij}V_{is}^*V_{js} + \delta_{ij}\delta_{is}C_{8ij}\right) \ .
\end{align}
To have a hermitian interaction the coefficients need to fulfil the relation $C_{lij} = C_{lji}^*$.