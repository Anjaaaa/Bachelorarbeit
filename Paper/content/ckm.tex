\cite[Chapter 20]{Peskin}
\cite[Chapter 1.2.1]{Tevatron}
The origins of the flavour mixing go back to the 1960s, when Italian physicist Cabibbo resolved anomalies in weak decay data by proposing a flavour mixing of lefthanded down-type quarks. Later in 1973, Kobayaski and Maskawa extended this idea to three quark generation to explain CP violation. On a theoretical level the flavour mixing arises from the fact that the fermion mass eigenstates do not necessarily equal the flavour eigenstates. In the course of this chapter we introduce the Glashow-Weinberg-Salam Model of electroweak interaction and derive how the fermions gain their masses and why this leads to flavour mixing.

\section{The Higgs Mechanism}
We consider a complex scalar field $\Phi$ that interacts with itself through a potential
\begin{align}
	V(\Phi) = -\mu^2\Phi^\dagger\Phi + \frac{\lambda}{2}(\Phi^\dagger\Phi)^2 \ , \quad \mu^2 > 0 \ .
\end{align}
If $\lambda>0$, the minimum of this potential occurs at
\begin{align}
	\langle\Phi\rangle = \Phi_0 =  \sqrt{\frac{\mu^2}{\lambda}} \ .
\end{align}
This is the vacuum expectation value of $\Phi$.

\subsubsection{$SU(2)$}
We add a $SU(2)$ gauge field coupled to $\Phi$, so $\Phi$ is a doublet $(\Phi_1,\Phi_2)$ with covariant derivative
\begin{align}
	D_\mu\Phi = (\partial_\mu - ig\sum_{a=1}^3A_\mu^a\tau^a)\Phi \ ,
\end{align}
where $\tau^a$ are the generators of $SU(2)$. In this case there is an infinite number of vacuum expection values for $\Phi$ arranged in a circle. We are free to choose one, and we make the simple choice
\begin{align}
	\langle\Phi\rangle = \frac{1}{\sqrt{2}}\begin{pmatrix} 0 \\ v \end{pmatrix} \ .
\end{align}
Note that by choosing one vacuum value, we break the symmetry.


The kinetic energy of $\Phi$ is
\begin{align}
	(D_\mu\Phi)^2 = &\frac{1}{2}(\partial_\mu v)(\partial^\mu v)\notag \\
	- &ig(\partial^\mu\begin{pmatrix} 0 & v \end{pmatrix}) \left(\sum_{a=1}^3A_\mu^a\tau^a\begin{pmatrix} 0 \\ v \end{pmatrix}\right)\notag \\
	- &\frac{1}{2}g^2\begin{pmatrix} 0 & v \end{pmatrix}\sum_{a,b=1}^{3}\tau^a\tau^b\begin{pmatrix} 0 \\ v \end{pmatrix}A_\mu^aA^{b\mu} \ .
\end{align}
Using the relation $\{\tau^a,\tau^b\}=\sfrac{1}{2}\cdot\delta_{ab}$, we can simplify the last expression to get
\begin{align}
	- \frac{1}{2}g^2\begin{pmatrix} 0 & v \end{pmatrix}\sum_{a,b=1}^{3}\tau^a\tau^b\begin{pmatrix} 0 \\ v \end{pmatrix}A_\mu^aA^{b\mu} &= - \frac{g^2v^2}{8}\sum_{a=1}^{3}A_\mu^aA^{a\mu} \ ,
\end{align}
which is a mass term $\mathcal{L}_m = -\frac{1}{2}m_A^2A_\mu A^\mu$ that assigns the mass $m_A = \frac{gv}{2}$ to all three gauge bosons.

\subsubsection{$SU(2)\times U(1)$}
We expand the system with an additional $U(1)$ symmetry with gauge boson $B$. The field $\Phi$ has a charge $Y_\Phi$ under $U(1)$. The new covariant derivative is
\begin{align}
	D_\mu\Phi = \left(\partial_\mu - ig\sum_{a=1}^3A_\mu^a\tau^2 - iY_\Phi g'B_\mu\right)\Phi \ .
\end{align}
Again, we examine the kinetic term
\begin{align}
	(D_\mu\Phi)^2 = &\frac{1}{2}(\partial_\mu v)(\partial^\mu v)\notag \\
	- &\frac{1}{2}i\left(\partial^\mu\begin{pmatrix} 0 & v \end{pmatrix}\right) \left(g\sum_{a=1}^3A_\mu^a\tau^2 + Y_\Phi g'B_\mu\right)\begin{pmatrix} 0 \\ v \end{pmatrix}\notag \\
	- &\frac{1}{2}\begin{pmatrix} 0 & v \end{pmatrix}
	\left(g^2\sum_{a,b=1}^3A_\mu^aA^{b\mu}\tau^a\tau^b + 2gg'Y_\Phi\sum_{a=1}^3A_\mu^a\tau^aB^\mu + Y_\Phi^2g'^2B_\mu B^\mu\right)
	\begin{pmatrix} 0 \\ v \end{pmatrix}
\end{align}
Using $\{\tau^a,\tau^b\}=\sfrac{1}{2}\cdot\delta_{ab}$ and replacing $\tau^a=\sfrac{\sigma^a}{2}$, we find for the last term
\begin{align}
	\mathcal{L}^{(\text{mass})} &= - \frac{v^2}{2}\left(g^2\frac{1}{4}\sum_{a=1}^3A_\mu^aA^{a\mu} + Y_\Phi^2g'^2B_\mu B^\mu - gg'Y_\Phi B^\mu A_\mu^3\right)\notag \\
	&= - \frac{1}{2}\frac{v^2}{4}\left(g^2A_\mu^1A^{1\mu} + g^2A_\mu^2A^{2\mu} + (gA_\mu^3 - 2g'Y_\Phi B_\mu)^2\right)\notag \\
	&= - \frac{1}{2}\frac{v^2}{4}\left(g^22W^+W^- + (g^2+4g'^2Y_\Phi^2)Z_0^2\right) \ .
\end{align}
Here we identified the known vector bosons
\begin{align}
	W_\mu^\pm &= \frac{1}{\sqrt{2}}(A_\mu^1\mp iA_\mu^2) \ , &&\quad &m_W &= \frac{v}{2}g \notag \\
	Z_\mu^0 &= \frac{1}{\sqrt{g^2+4g'^2Y_\Phi^2}}(gA_\mu^3 - 2g'Y_\Phi B_\mu) \ , &&\quad &m_Z &= \frac{v}{2}\sqrt{g^2+4g'^2Y_\Phi^2}
\end{align}
as mass eigenstates of the gauge bosons.


Since we are in a $SU(2)\times U(1)$ symmetry, there has to be a fourth gauge boson. As we have just derived, it is massless. We renounce giving an elaborate explanation for this, but we want to give a motivation. Therefore we need to remember that the masses arise through choosing a vacuum expectation value for $\Phi$ and thereby breaking the $SU(2)$ symmetry. Looking at the gauge transformation of $\Phi$
\begin{align}
	\Phi\rightarrow e^{i\sum_{a=1}^3\alpha^a\tau^a}e^{i\beta Y_\Phi}\Phi \ ,
\end{align}
we find that the choice $\alpha^1=\alpha^2=0$, $\alpha^3 = 2\beta Y_\Phi$ leaves the vacuum expectation value unchanged. Thus, parts of the symmetry are conserved and keep one gauge boson from acquiring mass. The fourth gauge boson is the photon, and it is orthogonal to $Z_\mu^0$:
\begin{align}
	A_\mu = \frac{1}{\sqrt{g^2 + 2g'^2Y_\Phi^2}}(2g'Y_\Phi A^3_\mu + gB_\mu) \ .
\end{align}

\section{Fermion Masses and Flavour Mixing}
In this and the next chapters we use the following notation for the chiral components of the particle functions. For the leptons:
\begin{align}
	E_R &= (e_R,\mu_R,\tau_R) \ , &&\quad &Y_E &= -1 \ ; \\
	L_L &= \left(\begin{pmatrix} \nu_e \\ e \end{pmatrix}_L,
	\begin{pmatrix} \nu_\mu \\ \mu \end{pmatrix}_L,
	\begin{pmatrix} \nu_\tau \\ \tau \end{pmatrix}_L\right) \ , &&\quad &Y_L &= -\frac{1}{2} \ ;
\end{align}
and accordingly for the quarks
\begin{align}
	U_R &= (u_R,c_R,t_R) \ , &&\quad &Y_U &= \frac{2}{3} \ ; \\
	D_R &= (d_R,s_R,b_R) \ , &&\quad &Y_D &= -\frac{1}{3} \ ; \\
	Q_L &= \left(\begin{pmatrix} u \\ d \end{pmatrix}_L,
	\begin{pmatrix} c \\ s \end{pmatrix}_L,
	\begin{pmatrix} t \\ b \end{pmatrix}_L\right) \ , &&\quad &Y_Q &= \frac{1}{6} \ .	
\end{align}
The charge under $U(1)_Y$, also hypercharge, is $Y$. It is related to the electric charge $Q$ and the third component of the weak isospin $I_3$ through the Gell-Mann–Nishijima formula: $Y = 2(Q-I_3)$. The righthanded particles are singlets under $SU(2)$ and therefore $I_3=0$. We will use as well $E_L = (e_L,\mu_L,\tau_L)$, $D_L = (d_L,s_L,b_L)$, and $U_L = (u_L,c_L,t_L)$.

\todo{Die Ys nachschlagen.}


The electroweak interaction lagrangian for the standard model fermions is
\begin{align}\label{eq:LFermion}
%	\mathcal{L}^{(\text{int})} = &\sum_{i=1}^{3}\bar{E}_R^i(i\slashed{\partial}-g_1Y_E\slashed{B})E_R^i + \bar{D}_R^i(i\slashed{\partial}-g_1Y_D\slashed{B})D_R^i + \bar{U}_R^i(i\slashed{\partial}-g_1Y_U\slashed{B})U_R^i \notag \\
%	+ &\sum_{i=1}^{3}\bar{L}_L^i(i\slashed{\partial}-g_1Y_L\slashed{B}-g_2\slashed{A})L_L^i + \bar{Q}_L^i(i\slashed{\partial}-g_1Y_Q\slashed{B}-g_2\slashed{A})Q_L^i \ ,
	\mathcal{L}^{(\text{int})} = &\bar{E}_R(i\slashed{\partial}-g_1Y_E\slashed{B})E_R + \bar{D}_R(i\slashed{\partial}-g_1Y_D\slashed{B})D_R + \bar{U}_R(i\slashed{\partial}-g_1Y_U\slashed{B})U_R \notag \\
	+ &\bar{L}_L(i\slashed{\partial}-g_1Y_L\slashed{B}-g_2\slashed{A})L_L + \bar{Q}_L(i\slashed{\partial}-g_1Y_Q\slashed{B}-g_2\slashed{A})Q_L \ .
\end{align}



The lagrangian in \eqref{eq:LFermion} describes massless particles. A fermion mass term couples the lefthanded and righthanded part of a particle. Since direct coupling between for example $e_R$ and $(\nu_e,e)_L$ would violate gauge invariance, a connecting field. To preserve invariance under Lorentz, $U(1)_Y$, and $SU(2)$ transformations this field must have spin 0, hypercharge $Y=\sfrac{1}{2}$, and be a doublet. We identify this field with $\Phi$ from the previous chapter and write the mass terms for the fermions
\begin{align}
%	\mathcal{L}^{(\text{mass})} = -\sum_{i,j=1}^{3}\left[\lambda^e_{ij} \bar{L}_L^i\Phi E_R^j  + \lambda_{ij}^d\bar{Q}_L^i\Phi D_R^j + \lambda_{ij}^u\sum_{a,b=1}^{2}\epsilon_{ab}\bar{Q}_{La}^{i}\Phi_b^\dagger U_R^j + \text{h.c.}\right] \ ,
	\mathcal{L}^{(\text{mass})} = -\left[\bar{L}_L\Phi\lambda^e  E_R  + \bar{Q}_L\Phi\lambda^d D_R + \bar{Q}_Li\sigma^2\Phi^\dagger\lambda^u U_R + \text{h.c.}\right] \ ,
\end{align}
with complex matrix coupling constants $\lambda^e,\lambda^d,\lambda^u$. Replacing $\Phi$ with its vacuum expectation value gives
\begin{align}\label{eq:LMass}
%	\mathcal{L}^{(\text{mass})} = -\frac{v}{\sqrt{2}}\sum_{i,j=1}^{3}\left[\lambda^e_{ij} \bar{E}_L^i E_R^j  + \lambda_{ij}^d\bar{D}_L^i D_R^j + \lambda_{ij}^u\bar{U}_L^{i} U_R^j + \text{h.c.}\right] \ ,
	\mathcal{L}^{(\text{mass})} = -\frac{v}{\sqrt{2}}\left[ \bar{E}_L\lambda^e E_R  + \bar{D}_L \lambda^dD_R + \bar{U}_L\lambda^u U_R + \text{h.c.}\right] \ .
\end{align}



The interaction lagrangian \eqref{eq:LFermion} is invariant under unitary transformations
\begin{align}
	E_L &\rightarrow S_eE_L && &E_R &\rightarrow R_eE_R \ , \\
	U_L &\rightarrow S_uU_L && &U_R &\rightarrow R_uU_R \ , \\
	D_L &\rightarrow S_dD_L && &D_R &\rightarrow R_dD_R \ .
\end{align}
Thus, we can diagonalize the interactions in \eqref{eq:LMass}. The diagonal lepton coupling is $\tilde{\lambda}^e = S_e\lambda^eR_e^\dagger$ and parametrizes the lepton masses
\begin{align}
	m_e = \frac{v}{\sqrt{2}}\tilde{\lambda}_{11}^e \ , \quad m_\mu = \frac{v}{\sqrt{2}}\tilde{\lambda}_{22}^e \ , \quad m_\tau = \frac{v}{\sqrt{2}}\tilde{\lambda}_{33}^e \ .
\end{align}
The diagonal coupling for up-type quarks is $\tilde{\lambda}^u = S_u\lambda^uR_u^\dagger$, giving the corresponding masses
\begin{align}
	m_u = \frac{v}{\sqrt{2}}\tilde{\lambda}_{11}^u \ , \quad m_c = \frac{v}{\sqrt{2}}\tilde{\lambda}_{22}^u \ , \quad m_t = \frac{v}{\sqrt{2}}\tilde{\lambda}_{33}^u \ .
\end{align}
The transformed coupling of the down-type quarks is $\tilde{\lambda}^d = S_d\lambda^dR_d^\dagger$, leading to the down-type masses
\begin{align}
	m_d = \frac{v}{\sqrt{2}}\tilde{\lambda}_{11}^d \ , \quad m_s = \frac{v}{\sqrt{2}}\tilde{\lambda}_{22}^d \ , \quad m_b = \frac{v}{\sqrt{2}}\tilde{\lambda}_{33}^d \ .
\end{align}


The transformed particle functions are now mass eigenstates. But when looking at couplings of up- and down-type quarks, e.g. the current
\begin{align}
	\bar{U}_L\gamma^\mu D_L \ ,
\end{align}
the transformation changes the interaction to
\begin{align}
	\bar{U}_L\gamma^\mu S_u^\dagger S_d D_L \ .
\end{align}
The matrix $V = S_u^\dagger S_d$ is the CKM matrix.