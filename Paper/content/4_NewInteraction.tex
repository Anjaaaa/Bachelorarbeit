In this chapter, we present an extension to the standard model proposed by Altmannshofer et. al. in publication \cite{InColour}. The authors originally aimed at explaining anomalies in the decay $B\rightarrow Kl\bar{l}$, but also obtained predictions for the direct detection of dark matter in the succeeding publication \cite{Z}. We will later compare their results with the formalism in the previous chapter that includes the CKM mixing.

\section{The New Interaction}
The extension to the standard model in \cite{InColour} is a new $U(1)'$ gauge group. The related vector-boson is called $Z'$, and it couples to the muon and tau lepton families, and a new set of vector-like quarks $U,D,Q$. The standard model quarks indirectly couple to the $Z'$ as well, since they mix with the new quarks through a Yukawa coupling:
\begin{align}
	\mathcal{L}^{(\text{mix})} = &\Phi' \bar{\tilde{D}}_R(Y_{Qb}b_L + Y_{Qs}s_L + Y_{Qd}d_L)\notag \\
	+ &\Phi'\bar{\tilde{U}}_R(Y_{Qt}t_L + Y_{Qc}c_L + Y_{Qu}u_L)\notag \\
	+ &\Phi'^\dagger\bar{U}_L(Y_{Ut}t_R + Y_{Uc}c_R + Y_{Uu}u_R)\notag \\
	+ &\Phi'^\dagger\bar{D}_L(Y_{Db}b_R + Y_{Ds}s_R + Y_{Dd}d_R) +\text{h.c.} \ ,
\end{align}
where $\tilde{Q}_R = (\tilde{U}_R,\tilde{D}_R)$, $Q_L = (U_L,D_L)$ are the weak doublets, $Y_{ij}$ are the coupling constants, and $\Phi'$ is a Higgs-like field that gives mass to the $Z'$.


In publication \cite{Z}, a coupling to a dark matter fermion $\chi$ is additionally established. The full interaction lagrangian then is:
\begin{align}
	\mathcal{L}^{(\text{int})}_{Z'} = &g'Z_\alpha'\times q_l\left(\bar{L}_L^2\gamma^\alpha L_L^2 - \bar{L}_L^3\gamma^\alpha L_L^3 + \bar{\mu}_R\gamma^\alpha\mu_R-\bar{\tau}_R\gamma^\alpha\tau_R\right)\notag \\
	+&g'Z_\alpha'\times v_\Phi^2\sum_{i,j}^3\left(-\frac{Y_{Di}Y^*_{Dj}}{2m_D^2}\bar{D}_R^i\gamma^\alpha D_R^j - \frac{Y_{Ui}Y^*_{Uj}}{2m_U^2}\bar{U}_R^i\gamma^\alpha U_R^j + \frac{Y_{Qi}Y_{Qj}^*}{2m_Q^2}\bar{Q}_L^i\gamma^\alpha Q_L^j\right)\notag \\
	+&g'Z_\alpha'\times q_\chi(\bar{\chi}\gamma^\alpha\chi) \ ,
\end{align}
where $q_l,q_\chi$ are the $U(1)'$ charge of the leptons and the dark matter particle, $m_{U,D,Q}$ are the masses of the new quarks, and $v_{\Phi'}$ is the vacuum expectation value of $\Phi'$.


\todo{Kann man tatsächlich hier die Ups drehen und dann später mit gedrehten Downs rechnen?}

\section{Restrictions on the Parameter Space\label{sec:ParamSpace}}
In \cite{Z} Altmannshofer et. al. discuss restrictions on the parameter space by looking at the $B$ decay mentioned above and discussing dark matter relic density and direct detection. They find that experimental data from $B\rightarrow Kl\bar{l}$ limits the ratio of the $Z'$ mass and the coupling $g'$ to
\begin{align}\label{eq:BoundBS}
	\SI{540}{\giga\electronvolt}\lessapprox\frac{m_{Z'}}{g'}\lessapprox\SI{4.9}{\tera\electronvolt} \ ,
\end{align}
with $m_{Z'}\gtrapprox\SI{10}{\giga\electronvolt}$.


\begin{minipage}{0.67\textwidth}
	Regarding the dark matter relic density, they conclude that only
	\begin{align}\label{eq:Relic}
	m_{Z'}\approx 2m_\chi
	\end{align}
	leads to correct results. Since they neglect flavour mixing in the nucleus, direct detection has to occur through the loop diagram in Figure \ref{fig:Loop}. The corresponding cross section at zero momentum transfer is
	\begin{align}\label{eq:Loop}
	\sigma_{0,\text{loop}} = \frac{\mu_{A\chi}^2}{A^2\pi}\left(\frac{\alpha_{em}Z}{3\pi}\ \frac{g'^2q_\chi q_l}{m_{Z'}^2}\log\left(\frac{m_\mu^2}{m_\tau^2}\right)\right)^2 \ ,
	\end{align}
	where $\mu_{A\chi}$ is the reduced mass of the nucleus and the dark matter particle $\chi$, and $A,Z$ are the nucleon and proton numbers.
\end{minipage} \hfill
\begin{minipage}{0.28\textwidth}
	\begin{figure}[H]
		\centering
		\resizebox{\textwidth}{!}{
			\begin{tikzpicture}
\tikzstyle{centerArrow}=[decoration={
	markings,
	mark=at position 0.5 with {\fill (2pt,0)--(-2pt,2.31pt)--(-2pt,-2.31pt)--cycle;}}]

\begin{scope}[xshift=6cm,yshift=5cm]
\def\xmove{2}
\def\ymove{1.25}
\def\centerShift{2cm}
\def\centerCircle{1cm}
\def\centerSize{0.05cm}
\coordinate (tCenter1) at (0,0);
\coordinate (tCenter2) at (0,-\centerShift);
\coordinate (tCenter3) at (0,-\centerShift-\centerCircle);
\coordinate (tCenter4) at (0,-\centerShift-\centerCircle-\centerShift);

\node (upperLeft) at (-\xmove,\ymove) {$\chi$};
\node (upperRight) at (\xmove,\ymove) {$\chi$};
\node (lowerLeft) at (-\xmove,-\centerShift-\centerCircle-\centerShift-\ymove cm) {$N$};
\node (lowerRight) at (\xmove,-\centerShift-\centerCircle-\centerShift-\ymove cm) {$N$};

\draw [centerArrow,postaction={decorate}]  (upperLeft) -- (tCenter1) ;
\draw [centerArrow,postaction={decorate}]  (tCenter1) -- (upperRight) ;
\draw [centerArrow,postaction={decorate}]  (lowerLeft) -- (tCenter4) ;
\draw [centerArrow,postaction={decorate}]  (tCenter4) -- (lowerRight) ;
\draw [decoration={snake, segment length=1.5mm, amplitude=0.5mm},decorate] (tCenter1) -- (tCenter2) ;
\node at (0.5,-\centerShift/2) {$Z'$};
\draw [
        decoration={markings, mark=at position 0.5 with {\fill (2pt,0)--(-2pt,2.31pt)--(-2pt,-2.31pt)--cycle;}, mark=at position 1 with {\fill (2pt,0)--(-2pt,2.31pt)--(-2pt,-2.31pt)--cycle;}},
        postaction={decorate}
] ([yshift=.5cm]tCenter3) ellipse(.45 and 0.5);
\node at ([yshift=-\centerCircle/2,xshift=0.75cm]tCenter2) {$l$};
\node at ([yshift=-\centerCircle/2,xshift=-0.75cm]tCenter2) {$l$};
\draw [decoration={snake, segment length=1.5mm, amplitude=0.5mm},decorate] (tCenter3) -- (tCenter4) ;
\node at (0.5,-\centerShift/2-\centerCircle-\centerShift) {$\gamma$};
\end{scope}
\end{tikzpicture}	
		}
		\captionsetup{width=\textwidth}
		\caption{Direct detection loop diagram}
		\label{fig:Loop}
	\end{figure}
\end{minipage}
%Regarding the dark matter relic density, they conclude that only
%\begin{align}\label{eq:Relic}
%	m_{Z'}\approx 2m_\chi
%\end{align}
%leads to correct results. Since they neglect flavour mixing in the nucleus, direct detection has to occur through the loop diagram in figure \ref{fig:Loop}. The corresponding cross section at zero momentum transfer is
%\begin{align}\label{eq:Loop}
%	\sigma_{0,\text{loop}} = \frac{\mu_{A\chi}^2}{A^2\pi}\left(\frac{\alpha_{em}Z}{3\pi}\ \frac{g'^2q_\chi q_l}{m_{Z'}^2}\log\left(\frac{m_\mu^2}{m_\tau^2}\right)\right)^2 \ ,
%\end{align}
%where $\mu_{A\chi}$ is the reduced mass of the nucleus and the dark matter particle $\chi$ and $A,Z$ are the nucleon and proton numbers.
%
%\begin{figure}[h!]
%	\centering
%	\begin{tikzpicture}
\tikzstyle{centerArrow}=[decoration={
	markings,
	mark=at position 0.5 with {\fill (2pt,0)--(-2pt,2.31pt)--(-2pt,-2.31pt)--cycle;}}]

\begin{scope}[xshift=6cm,yshift=5cm]
\def\xmove{2}
\def\ymove{1.25}
\def\centerShift{2cm}
\def\centerCircle{1cm}
\def\centerSize{0.05cm}
\coordinate (tCenter1) at (0,0);
\coordinate (tCenter2) at (0,-\centerShift);
\coordinate (tCenter3) at (0,-\centerShift-\centerCircle);
\coordinate (tCenter4) at (0,-\centerShift-\centerCircle-\centerShift);

\node (upperLeft) at (-\xmove,\ymove) {$\chi$};
\node (upperRight) at (\xmove,\ymove) {$\chi$};
\node (lowerLeft) at (-\xmove,-\centerShift-\centerCircle-\centerShift-\ymove cm) {$N$};
\node (lowerRight) at (\xmove,-\centerShift-\centerCircle-\centerShift-\ymove cm) {$N$};

\draw [centerArrow,postaction={decorate}]  (upperLeft) -- (tCenter1) ;
\draw [centerArrow,postaction={decorate}]  (tCenter1) -- (upperRight) ;
\draw [centerArrow,postaction={decorate}]  (lowerLeft) -- (tCenter4) ;
\draw [centerArrow,postaction={decorate}]  (tCenter4) -- (lowerRight) ;
\draw [decoration={snake, segment length=1.5mm, amplitude=0.5mm},decorate] (tCenter1) -- (tCenter2) ;
\node at (0.5,-\centerShift/2) {$Z'$};
\draw [
        decoration={markings, mark=at position 0.5 with {\fill (2pt,0)--(-2pt,2.31pt)--(-2pt,-2.31pt)--cycle;}, mark=at position 1 with {\fill (2pt,0)--(-2pt,2.31pt)--(-2pt,-2.31pt)--cycle;}},
        postaction={decorate}
] ([yshift=.5cm]tCenter3) ellipse(.45 and 0.5);
\node at ([yshift=-\centerCircle/2,xshift=0.75cm]tCenter2) {$l$};
\node at ([yshift=-\centerCircle/2,xshift=-0.75cm]tCenter2) {$l$};
\draw [decoration={snake, segment length=1.5mm, amplitude=0.5mm},decorate] (tCenter3) -- (tCenter4) ;
\node at (0.5,-\centerShift/2-\centerCircle-\centerShift) {$\gamma$};
\end{scope}
\end{tikzpicture}
%	\caption{Direct detection loop diagram}
%	\label{fig:Loop}
%\end{figure}


When discussing limits to the parameter space they distinguish two cases. For $q_l=q_\chi=1$, experimental data favours the parameter region
\begin{align}\label{eq:Boundg}
	\SI{10}{\giga\electronvolt}\lessapprox m_{Z'}\lessapprox\SI{46}{\giga\electronvolt}\notag \\
	\SI{2e-3}{}\lessapprox g'\lessapprox\SI{e-2}{} \ ,
\end{align}
leaving possible dark matter masses in the range $(5-23)\si{\giga\electronvolt}$. For $q_l=1,q_\chi=\sfrac{1}{6}$ no further restriction of the parameters can be found.

\todo{Evtl. kann man den Ursprung der Grenzen noch näher erläutern.}