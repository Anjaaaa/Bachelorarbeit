\begin{frame}{Dunkle Materie}
	\begin{figure}
	\centering
	\resizebox{.8\textwidth}{!}{
	\begin{tikzpicture}
		\pie[
		color={yellow, green, cyan},
		radius=2.5,
		text=legend,
%		explode={0,0,0}
		]{4.9/Gewöhnliche Materie, 26.8/ Dunkle Materie, 68.3/ Dunkle Energie}
	\end{tikzpicture}}
	\caption{Energieverteilung im Universum (ESA, Planck Teleskop 2013)}
\end{figure}
\end{frame}

\begin{frame}{Direct Detection}
	\begin{figure}[H]
		\resizebox{.6\textwidth}{!}{
			\begin{tikzpicture}
\tikzstyle{centerArrow}=[decoration={
    markings,
    mark=at position 0.5 with {\fill (2pt,0)--(-2pt,2.31pt)--(-2pt,-2.31pt)--cycle;}}]
\begin{scope}
\def\xmove{2.5}
\def\ymove{1.25}
\def\centerSize{0.15}
\node [fill, circle,inner sep=\centerSize cm] (tCenter)  {};
\node (upperLeft) at (-\xmove,\ymove) {$\chi$};
\node (upperRight) at (\xmove,\ymove) {$\chi$};
\node (lowerLeft) at (-\xmove,-\ymove) {$N$};
\node (lowerRight) at (\xmove,-\ymove) {$N$};
\draw [centerArrow,postaction={decorate}]  (upperLeft) -- (tCenter) ;
\draw [centerArrow,postaction={decorate}]  (lowerLeft) -- (tCenter) ;
\draw [centerArrow,postaction={decorate}]  (tCenter) -- (upperRight) ;
\draw [centerArrow,postaction={decorate}]  (tCenter) -- (lowerRight) ;
\end{scope}
\end{tikzpicture}
		}
		\caption{Streuung eines DM-Teilchens am Atomkern.}
	\end{figure}
\end{frame}